\section{Structures}
A (three-valued) structure with name \code{MyStruct} over a vocabulary \code{MyVoc} is declared by
\begin{lstlisting}
	structure MyStruct: MyVoc {
		//contents of the structures
	}
\end{lstlisting}
or by 
\begin{lstlisting}
	factlist MyStruct: MyVoc {
		//contents of the structures
	}
\end{lstlisting}

%%%%%%%%%%%%%%%%%%%%%%%%%%%%%%%%%%%%%%%%%%%%%%%%%%%%%%%%%
\subsection{Contents of a structure}
%%%%%%%%%%%%%%%%%%%%%%%%%%%%%%%%%%%%%%%%%%%%%%%%%%%%%%%%%


A particular input to a problem can be given by giving a (three valued) interpretation to all types and some predicate and function symbols of a given vocabulary. Here, we describe the different ways to specify a structure.

\subsubsection{Type Enumeration}
The syntax for a type enumeration is 
\begin{lstlisting}
	MyType = { El_1; El_2; ... ; El_n }
\end{lstlisting}
where {\tt MyType} is the name of the enumerated type and \code{El\_1; El\_2; ... ; El\_n } are the names of the objects of that type. Names of objects can be (positive and negative) integers, strings, chars, compound domain elements, or identifiers that start with an upper- or lowercase letter. 
Identifiers are shorthands for strings (without the quotes) and can be interchanged within \idp specifications. In lua-blocks, however, only the string variant should be used.

If one type is a subtype of another, all elements of the subtype are added to the supertype also.  In the case all subtypes of a given type are specified, the supertype is derived to be the union of all elements of the subtypes.  If a type is not specified, all domain elements of that type that occur in a predicate or function interpretation (see below) are automatically added to that type.  


\subsubsection{Predicate Enumeration}
The syntax for enumerating all tuples for which a predicate {\tt MyPred} with $n$ arguments is true is as follows.
\begin{lstlisting}
	MyPred = { El_1_1,..., El_1_n; 
				  ... ; 
				  El_m_1,..., El_m_n 
				}
\end{lstlisting}

It is also possible to write parentheses around tuples.
\begin{lstlisting}
	MyPred = { (El_1_1,..., El_1_n); 
				  ... ; 
				  (El_m_1,..., El_m_n) 
				}
\end{lstlisting}
This notation makes it possible to state that a proposition (a predicate with no arguments) is true, by using an empty tuple.
\begin{lstlisting}
	true = { () }
	false = { }
\end{lstlisting}
However, we recommend using \code{true} and \code{false} instead of \code{\{ () \}} and \code{\{\}}.

\subsubsection{Function Enumeration}
The syntax for enumerating a function {\tt MyFunc} with $n$ arguments is 
\begin{lstlisting}
	MyFunc = { El_1_1,...,El_1_n -> El_1;
		...;
		El_m_1,...,El_m_n -> El_m
	}
\end{lstlisting}
To give the interpretation of a constant, one can simply write `{\tt MyConst = El}' instead of `{\tt MyConst = \{ -> El \}}'.

% \subsubsection{Compound Domain Elements}
% A function applied to a tuple of domain elements can be used as a domain element.  We call such a domain element a \textit{compound domain element}.  An example is the domain element \code{F(1,a)}. If \code{F/n} is a function then 
% \begin{lstlisting}
% 	F = generate
% \end{lstlisting}
% specifies that the interpretation of \code{F} is the two-valued interpretation that maps each tuple \code{(d\_1,\ldots,d\_n)} to the compound domain element \code{F(d\_1,\ldots,d\_n)}.

\subsubsection{Three-Valued Predicate/Function interpretations}
Three-valued interpretations are given by either
\begin{itemize}
	\item enumerating the certainly true and certainly false tuples;
	\item enumerating the certainly true and the unknown tuples;
	\item enumerating the unknown and the certainly false tuples.
\end{itemize}
The third set of tuples can than be derived from the two that were given.
To specify which tuples are enumerated, use \code{<ct>}, \code{<cf>} and \code{<u>}.  For example
\begin{lstlisting}
	P<ct> = { /* enumeration of the certainly true tuples of P */ }
	P<u> = { /* enumeration of the unknown tuples of P */ }
\end{lstlisting}

\subsubsection{Interpretation by Procedures}
The syntax 
\begin{lstlisting}
	P = procedure MyProc
\end{lstlisting}
is used to interpret a predicate or function symbol \code{P} by a procedure \code{MyProc} (see below).  If \code{P} is an $n$-ary predicate, then \code{MyProc} should be an $n$-ary procedure that returns a boolean.  If \code{P} is an $n$-ary function, then \code{MyProc} should be and $n$-ary function that returns a number, string, or compound domain element (depending on the return type of \code{P}).


\subsubsection{Shorthands}
Shorthands like `{\tt MyType = \{1..10; 15..20\}}' or `{\tt MyType = \{ a..e; A..E \}}' may be used for enumerating types or predicates with only one argument.

%%%%%%%%%%%%%%%%%%%%%%%%%%%%%%%%%%%%%%%%%%%%%%%%%%%%%%%%%
\subsection{factlist}
%%%%%%%%%%%%%%%%%%%%%%%%%%%%%%%%%%%%%%%%%%%%%%%%%%%%%%%%%
A factlist consists of a list of facts in the usual syntax \code{symbolname(list of domain elements).}.
Semantically, a factlist is the structure which:
\begin{itemize}
\item All facts are true in the structure.
\item All other atoms over symbols which occur at least once in the factlist are \textbf{false}.
\end{itemize}
Currently, three-valued structures, structures which are partially given and partially defined (as in standard ASP for example) and function interpretations cannot be specified.

