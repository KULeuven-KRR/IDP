\documentclass{article}

\usepackage{xspace}
\usepackage{listings}
\usepackage{url}
\usepackage[usenames,dvipsnames]{color}
\usepackage{amssymb, amsmath}

\newcommand{\code}[1]{{\tt #1}}
\newcommand{\todo}[1]{\textcolor{blue}{TODO: #1}}
\newcommand{\ignore}[1]{}

\renewcommand{\bold}[1]{\textbf{#1}}

%% Logics taxonomy
\newcommand{\idp}{{\sc IDP}\xspace}
\newcommand{\idptwo}{{\sc IDP}$^2$\xspace}
\newcommand{\idpthree}{{\sc IDP}$^3$\xspace}
\newcommand{\minisatid}{{\sc MiniSAT(ID)}\xspace}
\newcommand{\gidl}{{\sc GidL}\xspace}
\newcommand{\fodotidp}{{\sc FO($\cdot$)$^{\mathtt{\tiny IDP}}$}\xspace}
\newcommand{\foidp}{\fodotidp}
\newcommand{\fodot}{{\sc FO($\cdot$)}\xspace}

%% Logic
% lor, land, lnot
\newcommand{\bigand}{\ensuremath{\bigwedge}}
\newcommand{\bigor}{\ensuremath{\bigor}}
\newcommand{\limplies}{\ensuremath{\Rightarrow}}
\newcommand{\limpliedby}{\ensuremath{\Leftarrow}}
\newcommand{\lequiv}{\ensuremath{\Leftrightarrow}}
\newcommand{\lrule}{\ensuremath{\leftarrow}}

\newcommand{\voc}{\ensuremath{\Sigma}}
\newcommand{\invoc}{\ensuremath{\sigma_{in}}}
\newcommand{\outvoc}{\ensuremath{\sigma_{out}}}
\newcommand{\instruct}{\ensuremath{J_{in}}}
\newcommand{\outstruct}{\ensuremath{J_{out}}}
\newcommand{\theory}{\ensuremath{T}\xspace}

%% TWO-VALUED
\newcommand{\true}{\ensuremath{\top}\xspace}
\newcommand{\false}{\ensuremath{\bot}\xspace}

%% THREE-VALUED
\newcommand{\ltrue}{\textbf{t}}
\newcommand{\lfalse}{\textbf{f}}
\newcommand{\lunkn}{\textbf{u}}

\newcommand{\atom}{\ensuremath{a}\xspace}
\newcommand{\lit}{\ensuremath{l}\xspace}

\newcommand{\rules}{\ensuremath{R}\xspace}
\newcommand{\defined}[1]{\ensuremath{#1_{def}}\xspace}
\newcommand{\open}[1]{\ensuremath{#1_{open}}\xspace}

\newcommand{\xxx}{\ensuremath{\overline{x}}}
\newcommand{\yyy}{\ensuremath{\overline{y}}}
\newcommand{\zzz}{\ensuremath{\overline{z}}}
\newcommand{\ddd}{\ensuremath{\overline{d}}}
\newcommand{\ttt}{\ensuremath{\overline{t}}}

\renewcommand{\|}{\ensuremath{\,|\,}}
\newcommand{\elim}{\ensuremath{\backslash}}

\newcommand{\subs}{\ensuremath{/}}

\newcommand{\func}[1]{\ensuremath{f(#1)}\xspace}

\newcommand{\set}{\ensuremath{S}\xspace}
\newcommand{\inter}{\ensuremath{I}\xspace}
\newcommand{\partinter}{\ensuremath{J}\xspace}

\newcommand{\fone}{\ensuremath{\varphi}\xspace}
\newcommand{\ftwo}{\ensuremath{\psi}\xspace}


%% Inferences
\newcommand{\mx}[3]{\ensuremath{<#1, #2, #3>}}

%% Source code fragments
\usepackage{listings}

\lstdefinelanguage{idp}{
	morekeywords=[1]{namespace,vocabulary,theory,structure,procedure},
	morekeywords=[2]{include,using,type,isa,contains,partial,extern,LFD,GFD},
	morekeywords=[3]{int,float,char,string,nat},
	morecomment=[s]{/*}{*/},	
	morecomment=[l]{//}
}
\lstset{
	language=idp,
	tabsize=3,
	frame=none,
	basicstyle=\ttfamily,
	commentstyle=\color{Gray},
	keywordstyle=[1]\color{BrickRed}\bfseries,
	keywordstyle=[2]\color{OliveGreen}\bfseries,
	keywordstyle=[3]\color{Blue}\bfseries,
}


\title{Redefining \fodotidp}
\author{Broes~De~Cat}

\begin{document}
\maketitle

Een lijst van problemen of onduidelijkheden in de huidige taal, zo duidelijk mogelijk geformuleerd, liefst met voorbeeldjes.
\todo{Vermoedelijk mag hier enige vorm van orde in gestoken worden}



\section{Constants for scripting}
Het is nu mogelijk placeholders te schrijven van de vorm $ \$n$. Dit zijn placeholders voor domeinelementen, die dan \emph{op commandline} moeten worden meegegeven om te instanti\"eren.

In de oude grounder was dit noodzakelijk om makkelijk te kunnen scripten, in de nieuwe grounder lijkt mij dit overkill aangezien we dit perfect kunnen emuleren met lua syntax. Het is nu wel mogelijk om dit op andere plaatsen te gebruiken dan enkel in structuren, maar dat kom sowieso de duidelijkheid niet ten goede.

Voorstel: afschaffen 



\section{Lua-logic syntax}
De manier waarop momenteel naar logische constructies gerefereerd kan worden is op zijn minst awkward en moet eigenlijk helemaal herzien worden.

Wat zijn momenteel de logische constructies en hoe wordt ernaar gerefereerd?
Veronderstel structuur \code{S}, voc \code{V}, theorie \code{T}, type \code{t}, predicaat \code{p}, predtable \code{pt}, sorttable \code{st}
\begin{table}[!htp]
\centering
\begin{tabular}{c|c}
constructie & gebruik \\
\hline
type selecteren in theorie, bvb \code{MAX}, \code{MIN} & \code{MAX[:t]} \\
predicaat uit voc & \code{V[p]} \\
type uit voc & \code{V[t.type]} \\
predtabel uit voc & \code{V[p]} \\
sorttabel uit voc & \code{S[V::t.type]} \\
predtabel van sort uit voc & \code{S[V::t.pred]} \\
& En error voor \code{S[V::t]} \\
ct van predtabel & \code{pt.ct} \\
type van pred positie 1 & ? \\
formula uit theorie & ? \\
\ldots & \ldots
\hline
\end{tabular}
\end{table}


\section{Names for \fodot constructs}
hier zou best een correcte, formele definitie komen van \fodot, bruikbaar in al onze papers (met weglating van niet-relevante delen).


\section{Typing}

\subsection{Standard types}
Zit er een hierarchie in de standaard types zelf? Geldt er \code{int isa float} en \code{nat isa int}? Is een char een string?

\subsubsection{Range definitions in structure}
Verwant probleem, wat kan je in een structuur schrijven van welk type? 
Momenteel zijn er ranges van ints en van chars.
Het is contra-intuitief om een intrange te schrijven in een float type of in een string type (maar het mag dus wel).

Doubles en strings kunnen (naast de int en char-ranges) enkel geenumereerd worden. Dit is ook vreemd als een sort wel over het oneindige type kan lopen.

Voorstel:
intranges bij ints, charranges bij chars, doubleranges bij doubles en enkel enumeraties bij strings.
Verliezen we dan iets? Nee, want je kan lua schrijven om bvb strings te genereren. 

\subsection{Overloading}
Hoe lossen we het overloading <=> type derivatie probleem op? Momenteel is het zo geimplementeerd dat in geval van overloading, de gebruiker elk type moet specifieren.

\subsection{Subtyping en semantics}
Hoe zit het juist met predicaat introductie bij subtyping? Er zijn een aantal dingen die niet consistent of niet doorzichtig worden ge\"implementeerd, waardoor het voor de gebruiker bijna onmogelijk is om te weten wat er zal gebeuren.

\[\forall x: P(x+1)\]
heeft een andere betekenis dan
\[\{\forall x: P(x+1) \lrule \true \}\]
Reden: er wordt een andere manier van predicaat-introductie gedaan.

Verder, \[\{P(C) \lrule \true \}\] heeft een modellen als $C$ niet afbeeldt op een element uit het type van $P$. Dit is erg contra-intuitief, want als gebruiker heb je geasserteerd dat $P(C)$ waar moet zijn.

\subsubsection{Definition semantics}
Hierop verder gaand, zou het soms nuttig zijn om als gebruiker te speficieren dat je een concept gedefinieerd op een oneindig type wilt beperken door het uit te rekenen alsof het op het oneindige type was, maar dan de projectie te nemen op het eindige interval.

\begin{lstlisting}
vocabulary V {
  type Number isa nat
  Even(Number)
}
theory T1 : V {
  { Even(0).
    Even(x) <- ~Even(x-1). }
}
theory T2 : V {
  { Even(0).
    Even(x+1) <- ~Even(x). }
}
structure S : V {
  Number = { 1..5 } // NOTE: bevat 0 NIET
}
}
\end{lstlisting}


\section{UNA}
Vroeger was het mogelijk om in een vocabularium te schrijven dat je unique names had en welke constanten (en overeenkomstige domeinelementen dat waren).
De syntax hiervoor was \code{type X = {A;B;C;D}}, waarbij een type $X$ werd aangemaakt dat bestond uit de 4 constanten.
Dat was een on-logische manier van werken, maar een die praktisch wel erg handig was.

Momenteel is het enkel mogelijk om eerst een type te maken, dan alle constanten expliciet apart te declareren over dat type, dan in de structuur het type te instantieren en dan nog eens elke constante apart een waarde te geven.

Omdat dat toch wel wat teveel van het goeie is, een nieuw voorstel:
In het vocabularium maken we een shorthand om een reeks constanten te declareren over 1 type, bvb de oude syntax.
Dan zou er in de theorie een statement kunnen bijkomen dat stelt dat UNA voor het type geldt en dat dus voor elke constante daarover een uniek domeinelement gemaakt wordt.

\todo{Zijn er hier subtyping problemen mogelijk?}  

\subsection{Defined types}
Er is ook nog het idee (grotendeels geimplementeerd behalve de syntaxbeslissing), over hoe je nieuwe types zou kunnen definieren in functie van uitrekenbare formules. Eigenlijk gaat het hier over functie-constructoren waarbij hun beeld nieuw geïntroduceerde domeinelementen zijn.

bvb \code{type tuple} in vocabularium en \code{{ tuple(x, y) <- Pos(x) \& Pos(y)}} in de structuur genereerd een type tuple dat bestaat uit alle ``domeinelementen'' \code{tuple(x, y)} voor elke $x$ en $y$ die posities zijn. Er zijn dan ook functies beschikbaar om uit een type zijn argumenten te extraheren (\todo{staan nog niet bij de syntaxlijst}).



\section{Numeric operations}
Hoe zit het juist met numerische operaties die eigenlijk overgaan in andere types? Integer deling, float equality, \ldots?



\section{Warnings - errors - styles}
Voor welke constructies in \foidp geven we een warning en waarvoor geven we een error? Het idee van \emph{styles} kunnen we hier best terug voor opnemen: een serie van parser/checker opties die aangeven wat toegelaten constructies zijn. Dan natuurlijk ook de verwante problemen met het includen van files die met een vrijere style geschreven zijn en dergelijke.



\section{Sets}
Momenteel zit er enige support voor sets in de taal: sets van de vorm $\{\xxx[\ttt]: \fone : \func{\xxx}\}$ en $[set1, set2, \ldots]$, waarbij $set_i$ ook sets zijn. De enige plaats waar die gebruikt kunnen worden zijn binnen aggregaatsfuncties.

Daarentegen is het eigenlijk wel mogelijk om die syntax algemener toe te laten (zie bvb zinc)
\begin{itemize}
  \item types die sets zijn
  \item quantificaties over sets en variabelen over sets
  \item expliciete operaties op sets: unie (vervangt bovenstaande ``enumset'') en eventueel doorsnede
\end{itemize}

Dit kan sterk de leesbaarheid van theorieen verhogen.

\subsection{Type specificatie via sets}
Een andere plaats waar we al enige tijd sets willen invoeren is bij typering
\[\exists x[x<3 \land P(x)]\] zou bvb betekenen: x ranges over alle domeinelementen kleiner dan 3 en waarvoor $P(x)$ true is.

\emph{Binarie quantoren} bereiken we daar ook meteen mee, want op die manier wordt het schrijven van $\forall x: P(x) \limplies Q(x)$ en $\exists x: P(x) \land Q(x)$ beide gereduceerd tot $\forall x[P(x)]: Q(x)$ en $\exists x[P(x)]: Q(x)$.



\section{Syntactic sugar}
Last but not least, er is heel wat syntactische suiker mogelijk (en nodig) om de taal beter bruikbaar te maken.

Voor het lua gedeelte van de taal is dit (op korte termijn al) te realiseren door extra procedures te implementeren die inter de oude syntax dan gebruiken.

Voor de rest van de taal zal het neerkomen op het implementeren van een preprocessor laag tussen de parser en het doorgeven van data aan het systeem.

Voorstellen?
\begin{itemize}
  \item \code{one}: $\exists_1$ (alloy)
  \item \code{lone}: $\exists_{<2}$ (alloy)
  \item \code{no}: $\lnot \exists$ (alloy)
  \item \code{\#\{P\}=3}: het totaal aantal tupels dat aan P voldoet is drie (shorthand voor \code{\#\{a b c d: P(a,b,c,d)\}=3}) (alloy)
  \item In structuur specifi\"eren dat een type $n$ elementen zou moeten hebben (maar welke maakt niet uit) (alloy)
  \item if-then-else blokken: momenteel is het nodig om de conditie dubbel te schrijven (zinc)
  \item sets (zie boven): te vertalen naar predicaten met gepaste ariteit
  \item lists/arrays: te vertalen naarpredicaten met een extra ``index'' argument
  \item tupels: te vertalen naar gedefinieerd type met gepaste functies om elementen op te vragen
\end{itemize}

\end{document}