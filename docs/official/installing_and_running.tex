\section{Installing And Running}
The system has been verified to run under Windows and various Unix versions.
The system also works under OSX, but for any version at or below Lion it requires developer components on the user machine to run.

\subsection{Getting the system}
\subsubsection{Downloading the most recent version}
Pre-built binaries can be retrieved from \url{http://dtai.cs.kuleuven.be/krr/software/idp} and installed in their default (OS-specific) way.
For the reasons stated above, we do not provide pre-built OSX packages. 

\subsubsection{Building from source}
Required software packages:
\begin{itemize}
  \item C and C++ compiler, supporting most of the C++11 standard. Examples are GCC 4.4, Clang 3.1 and Visual Studio 11 or higher.
  \item Cmake build environment. 
  \item Bison and Flex parser generator software.
  \item Pdflatex for building the documentation.
  \item (optional) Gecode for constraint programming support.
\end{itemize}

Assume idp is unpacked in \code{idpdir}, you want to build in \code{builddir} (cannot be the same as \code{idpdir}) and install in \code{installdir}. Building and installing is then achieved by executing the following commands:
\begin{lstlisting}
cd <builddir>
cmake <idpdir> -DCMAKE_INSTALL_PREFIX=<installdir> 
               -DCMAKE_BUILD_TYPE="Release"
make -j 4
make check
make install
\end{lstlisting}

Alternatively, cmake-gui can be used as a graphical way to set cmake options.

\subsection{Running the software}
\subsubsection{Batch mode}
One-shot execution of a procedure \code{proc} with a set of files \code{files} is achieved by running
\begin{lstlisting}
runidp -e "proc()" files
\end{lstlisting}
Omitting the \code{-e} option results in execution of the \code{main} method if any is available.

\subsubsection{Interactive mode}
An interactive session, with (optionally) a set of files \code{files}, is started with 
\begin{lstlisting}
runidp files -i
\end{lstlisting}

Afterwards, help can be requested with the \code{help()} command. 
Auto-completion of available commands is available via the tab key.
Additional files can be included with the \code{parse} command.
