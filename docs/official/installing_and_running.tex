
The system has been verified to run under Windows (7), OSX (Lion) and various unix versions.

\subsection{Building from source}
Required software packages:
\begin{itemize}
  \item C and C++ compiler, supporting most of the C++11 standard. Examples are GCC 4.4 or higher, clang 3.2 or visual studio 11.
  \item Cmake build environment. 
  \item Bison and flex packages or yacc and lex packages.
  \item Pdflatex and doxygen for building the documentation.
  \item (optional) Gecode for constraint programming support.
\end{itemize}

Assume idp is unpacked in \code{idpdir}, you want to build in \code{builddir} (cannot be the same as \code{idpdir}) and install in \code{installdir}. Building and installing is then achieved by executing the following commands:
\begin{lstlisting}
cd <builddir>
cmake <idpdir> -DCMAKE_INSTALL_PREFIX=<installdir> -DCMAKE_BUILD_TYPE="Release"
make -j 4
make check
make install
\end{lstlisting}

Alternatively, cmake-gui can be used as a graphical way to set cmake options.

\subsection{Running the software}
One-shot execution of a procedure \code{proc} with a set of files \code{files} is achieved by running
\begin{lstlisting}
idp -e "proc()" files
\end{lstlisting}
Omitting the \code{-e} option results in execution of the main method.

An interactive interface using the set of files \code{files} is started with 
\begin{lstlisting}
idp -i files
\end{lstlisting}

Afterwards, help can be requested with the \code{help()} command. Auto-completion is available via the tab key, ctrl-r will search the command history.

During an interactive session, it is possible to include additional files with the \code{parse} command.
