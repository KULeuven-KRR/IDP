\section{Vocabularies}
A vocabulary with name {\tt MyVoc} is declared by
\begin{lstlisting}
	vocabulary MyVoc {
		// contents of the vocabulary
	}
\end{lstlisting}
A vocabulary can contain symbol declarations, symbol pointers, and other vocabularies. Symbols are types (sorts), predicate and functions symbols.


%%%%%%%%%%%%%%%%%%%%%%%%%%%%%%%%%%%%%%%%%%%%%%%%%%%%%%%%%
\subsection{Symbol declarations}
%%%%%%%%%%%%%%%%%%%%%%%%%%%%%%%%%%%%%%%%%%%%%%%%%%%%%%%%%
\label{ssec:symbols}
A type with name \code{MyType} is declared by
\begin{lstlisting}
	type MyType
\end{lstlisting}
When declaring a type, it can be stated that this type is a subtype or supertype of a set of other types.  The following declares \code{MyType} to be a supbtype of the previously declared types \code{A1} and \code{A2}, and a supertype of the previously declared types \code{B1} and {B2}:
\begin{lstlisting}
	type MyType isa A1, A2 contains B1, B2
\end{lstlisting}
In the rest of this text, we will sometimes use ``parent type'' as direct supertype and ``ancestor type'' for (in)direct supertype.

A predicate with name \code{MyPred}, arity 3 and types \code{T1},\code{T2},\code{T3} is declared by
\begin{lstlisting}
	MyPred(T1,T2,T3)
\end{lstlisting}

A predicate with arity zero can be declared by \code{MyPred()} or \code{MyPred}.

A function with name \code{MyFunc}, input types \code{T1},\code{T2},\code{T3} and output type \code{T}  is declared by
\begin{lstlisting}
	MyFunc(T1,T2,T3):T
\end{lstlisting}

A partial function is declared by 
\begin{lstlisting}
	partial MyFunc(T1,T2,T3):T
\end{lstlisting}

Constants of type \code{T} can be declared by \code{MyConst:T} or \code{MyConst():T}.  Besides functinos with an identifier as name, functions of arity two with names \code{+,-,*,/,\%}  and \code{\textasciicircum} can be declared, as well as unary functions iwth names \code{-} and \code{abs}.

Any symbol has to be declared on its own line.



%%%%%%%%%%%%%%%%%%%%%%%%%%%%%%%%%%%%%%%%%%%%%%%%%%%%%%%%%
\subsection{Symbol pointers}
%%%%%%%%%%%%%%%%%%%%%%%%%%%%%%%%%%%%%%%%%%%%%%%%%%%%%%%%%

To include a type, predicate, or function from a previously declared vocabulary \code{V} in another vocabulary \code{W}, write
\begin{lstlisting}
	/* Declaration of vocabulary V*/
	vocabulary V {
		//...
		type A
		P(A)
		F(A,A):A
		//...
	}

	vocabulary W {
		extern type V::A
		extern V::P[A]		//also possible: extern V::P/1
		extern V::F[A,A:A]	//also possible: extern V::F/2:1
	}
\end{lstlisting}

\todo{verwijder using en extern, zie meeting. Het wordt extends, right?}

In the example, explicitly including type \code{A} of vocabulary \code{V} in \code{W} is not needed, since types of included predicates or functions are automatically included themselves.  To include the whole vocabulary \code{V} in \code{W} at once, used
\begin{lstlisting}
	vocabulary W {
		extern vocabulary V
	}
\end{lstlisting}




%%%%%%%%%%%%%%%%%%%%%%%%%%%%%%%%%%%%%%%%%%%%%%%%%%%%%%%%%
\subsection{The standard vocabulary}
%%%%%%%%%%%%%%%%%%%%%%%%%%%%%%%%%%%%%%%%%%%%%%%%%%%%%%%%%

The global namespace contains a fixed vocabulary \code{std}, which is defined as follows:

\begin{lstlisting}
	vocabulary std {
		type nat	
		type int contains nat
		type float contains int
		type char
		type string contains char

		+(int,int) : int
		-(int,int) : int
		*(int,int) : int
		/(int,int) : float
		%(int,int) : int
		abs(int) : int
		-(int) : int
		
		+(float,float) : float
		-(float,float) : float
		*(float,float) : float
		/(float,float) : float
		^(float,float) : float
		abs(float) : float
		-(float) : float
	}
\end{lstlisting}
Every vocabulary implicitly contains all symbols of \code{std}.  Also, every vocabulary contains for each of its types \code{A} the predicates \code{=(A,A)}, \code{<(A,A)}, and \code{>(A,A)} and the functions \code{MIN:A}, \code{MAX:A}, \code{SUCC(A):A} and \code{PRED(A):A}. In every structure, the symbols of \code{std} have the following interpretation:

\begin{center}
	
\begin{tabular}{l|l}
\code{nat} & all natural numbers \\
\code{int} & all integer numbers \\ 
\code{float} & all floating point numbers \\
\code{char} & all characters \\
\code{string} & all strings \\
\code{+(int,int) : int} & integer addition \\
\code{-(int,int) : int } &integer subtraction \\
\code{*(int,int) : int} & integer multiplication \\
\code{/(int,int) : float} & division \\
\code{\%(int,int) : int} & remainder \\
\code{abs(int) : int} & absolute value \\
\code{-(int) : int} & unary minus \\
\code{+(float,float) : float} & floating point addition \\
\code{-(float,float) : float} & floating point subtraction \\
\code{*(float,float) : float} & floating point multiplication \\
\code{/(float,float) : float} & floating point division \\ 
\code{\textasciicircum(float,float) : float} & floating point exponentiation \\
\code{abs(float) : float} & absolute value \\
\code{-(float) : float} & unary minus
\end{tabular} 

\end{center}

The predicate \code{=/2} is always interpreted by equality.  The order $<_{dom}$ on domain elements is defined by
\begin{itemize}
	\item numbers are smaller than non-numbers;
%\item strings are smaller than compound domain elements (see bolow for a definitions of a compound domain element);
\item $d_1<_{dom} d_2$ if $d_1$ and $d_2$ are numbers and $d_1<d_2$;
\item $d_1<_{dom} d_2$ if $d_1$ and $d_2$ are strings that are no numbers and $d_1$ is before $d_2$ in the lexicographic ordering;
\item $d_1<_{dom} d_2$ is some total order on compound domain elements (which we do not specify).
\end{itemize}

Every structure contains the following fixed interpretations:

\begin{center}
	
\begin{tabular}{l|l}
\code{<(A,A)} & the projection of $<_{dom}$ to the domain of A \\
\code{>(A,A)} & the projection of $>_{dom}$ to the domain of A \\
\code{MIN:A} & the $<_{dom}$-least element in the domain of A \\
\code{MAX:A} & the $<_{dom}$-greatest element in the domain of A \\
\code{SUCC(A):A} & the partial function that maps an element a  of the domain of \code{A} \\ & to the $<_{dom}$-least element of the domain of A that is strictly larger than a \\
\code{PRED(A):A} & the partial function that maps an element a  of the domain of \code{A} \\ & to the $<_{dom}$-greatest element of the domain of A that is strictly smaller than a \\
\end{tabular} 

\end{center}

In an IDP-file, you should disambiguate which \code{MAX} you want to use.  This is done by \code{MAX[:MyType]}.
