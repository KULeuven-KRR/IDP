
A namespace with name \code{MySpace} is declared by
\begin{lstlisting}
	namespace MySpace {
		// content of the namespace
	}
\end{lstlisting}

An object with name \code{MyName} declared in namespace \code{MySpace} can be referred to by \code{MySpace::MyName}.  Inside \code{MySpace}, \code{MyName} can simply be referred to by \code{MyName}.

A Namespace can contain namespaces, vocabularies, theories, structures, terms, queries, procedures, options and using statements.  A using statement is of one of the following forms
\begin{lstlisting}
	using namespace MySpace
	using vocabulary MyVoc
\end{lstlisting}
where \code{MySpace} is the name of a namespace, and \code{MyVoc} the name of a vocabulary.  Below such a using statement, objects \code{MyObj} declared in \code{MySpace}, respectively \code{MyVoc}, can be referred to by \code{MyObj}, instead of \code{MySpace::Myobj}, respectively \code{MyVoc::MyObj}.

Every object that is declared outside a namespace, is considered to be part of the global namespace.  The name of the global namespace is \code{global\_namespace}.  In other words, eery \idp file implicitely starts with \code{namespace global\_namespace \{} and ends wit han extra \code{\}}.

\todo{uitleg bij de nieuwe namespaces structuur}
\todo{naam van de global namespace}
\todo{using <mx>}