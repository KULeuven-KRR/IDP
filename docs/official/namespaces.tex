\section{Namespaces}
A namespace with name \code{MySpace} is declared by
\begin{lstlisting}
namespace MySpace {
	// content of the namespace
}
\end{lstlisting}

A Namespace can contain namespaces, vocabularies, theories, structures, terms, queries, procedures, options and using statements.  

An object with name \code{MyName} declared in namespace \code{MySpace} can be referred to by absolute qualification: \code{MySpace::MyName}.  
Inside \code{MySpace}, \code{MyName} can simply be referred to by \code{MyName}.
Additionally, \emph{using} statements can be used to allow relative qualification.
A using statement is of one of the following forms
\begin{lstlisting}
using namespace MySpace
using vocabulary MyVoc
\end{lstlisting}
where \code{MySpace} is the name of a namespace, and \code{MyVoc} the name of a vocabulary.  Below such a using statement, objects \code{MyObj} declared in \code{MySpace}, respectively \code{MyVoc}, can be referred to by \code{MyObj}, instead of \code{MySpace::Myobj}, respectively \code{MyVoc::MyObj}.

Every object that is declared outside a namespace, is considered to be part of the global namespace, called \code{\globalspace}.  
In other words, every \idp file implicitely starts with \code{namespace \globalspace \{} and ends with an additional \code{\}}.
Everything declared inside a library is contained within the namespace \code{\stdspace}. 
It is discouraged to add or overwrite objects within \stdspace. 

\todo{define object}