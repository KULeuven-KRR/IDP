\section{\idptwo vs \idpthree}
Users of the old \idptwo-system will still have a bunch of files with different syntaxis. Here are the basic rules for transforming them to \idpthree.
\begin{itemize}
	\item Blocks in the \idptwo system now are structures, vocabularies and theories.
		\begin{itemize}
			\item Everything that used to be in a Given: Declare: or Find: block, now belongs to a vocabulary.
			\item The sentences and definitions from the Satisfying: block should be moved to a theory.
			\item The Data: block is essentialy what is now a structure.
		\end{itemize}
	\item UNA-DCA declarations in the vocabulary are no longer allowed:
		\begin{itemize}
			\item In the \idptwo system you could write \code{type Direction={Up;Down}}. This had the effect of creating new domain elements Up and Down and constants Up and Down that could be used in the theory (Satisfying block).
			\item For the moment, this is not yet possible in the \idpthree system. If you want the same effect: you add \code{type Direction}, \code{Up: Direction} and \code{Down: Direction} to the vocabulary (this creates the type and makes the constants. In the structure, you interprete Direction by \code{Direction = {u;d}} and you interprete the constants by \code{Up = u; Down = d}.
		\end{itemize}
	\item Three-valued interpretations have a slightly different syntax.
		\begin{itemize}
			\item In a Data: block of the \idptwo system, one could write \code{P = {A;B}{C}}, meaning that \code{P} should be certainly true for \code{A} and \code{B} and that \code{P} should be certainly false for \code{C}. In \idpthree, we write (as explained above \code{ P<ct> = {A;B}} and \code{ P<cf> = {C}}.
		\end{itemize}
	\item Aggregates have a slightly different syntaxis (see above).


\end{itemize}