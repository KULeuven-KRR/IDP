\section{Procedures}
%%%%%%%%%%%%%%%%%%%%%%%%%%%%%%%%%%%%%%%%%%%%%%%%%%%%%%%%%
\subsection{Declaring a procedure}
%%%%%%%%%%%%%%%%%%%%%%%%%%%%%%%%%%%%%%%%%%%%%%%%%%%%%%%%%

A procedure with name {\tt MyProc} and arguments {\tt A1}, \ldots, {\tt An} is declared by 
\begin{lstlisting}
	procedure MyProc(A1,...,An) {
		// contents of the procedure
	}
\end{lstlisting}
Inside a procedure, any chunk of Lua code can be written. For Lua's reference manual, see \url{http://www.lua.org/manual/5.1/}. In the following, we assume that the reader is familiar with the basic concepts of Lua. Like in most programming languages, a procedure should be declared before it can be used in other procedures (either in the same file or in earlier included files). There is one exception to this: procedures in the global namespace (for example, all built-in procedures) can always be used, no matter what.

%%%%%%%%%%%%%%%%%%%%%%%%%%%%%%%%%%%%%%%%%%%%%%%%%%%%%%%%%
\subsection{\idp types}
%%%%%%%%%%%%%%%%%%%%%%%%%%%%%%%%%%%%%%%%%%%%%%%%%%%%%%%%%

Besides the standard types of variables available in Lua, the following extra types are available in \idp procedures.
\begin{description}
	\item[sort] A set of sorts with the same name. Can be used as a single sort if the set is a singleton.
	\item[predicate\_symbol] A set of predicates with the same name, but possibly with different arities. Can be used as a single predicate if the set is a singleton. If {\tt P} is a predicate\_symbol and {\tt n} an integer, then {\tt P/n} returns a predicate\_symbol containing all predicates in {\tt P} with arity {\tt n}. If {\tt s1}, \ldots, {\tt sn} are sorts, then {\tt P[s1,...,sn]} returns a predicate\_symbol containing all predicates $Q/n$ in {\tt P}, such that the $i$'th sort of $Q$ belongs to the set {\tt si}, for $1 \leq i \leq n$.
	\item[function\_symbol] A set of first-order functions with the same name, but possibly with different arities. Can be used as a single first-order function if the set is a singleton. If {\tt F} is a function\_symbol and {\tt n} an integer, then {\tt F/n:1} returns a function\_symbol containing all function in {\tt F} with arity {\tt n}. If {\tt s1}, \ldots, {\tt sn}, {\tt t} are sorts, then {\tt F[s1,...,sn:t]} returns a function\_symbol containing all functions $G/n$ in {\tt F}, such that the $i$'th sort of $F$ belongs to the set {\tt si}, for $1 \leq i \leq n$, and the output sort of $G$ belongs to {\tt t}.
	\item[symbol] A set of symbols of a vocabulary with the same name. Can be used as if it were a sort, predicate\_symbol, or function\_symbol.
	\item[vocabulary] A vocabulary. If {\tt V} is a vocabulary and {\tt s} a string, {\tt V[s]} returns the symbols in {\tt V} with name {\tt s}. 
	\item[compound] A domainelement of the form $F(d_1,\ldots,d_n)$, where $F$ is a first-order function and $d_1$, \ldots, $d_n$ are domain elements.
	\item[tuple] A tuple of domain elements. {\tt T[n]} returns the {\tt n}'th element in tuple {\tt T} (This is 1-based, thus the first element is referred to as {\tt T[1]}).
	\item[predicate\_table] A table of tuples of domain elements.
	\item[predicate\_interpretation] An interpretation for a predicate. If {\tt T} is a predicate\_interpretation, then {\tt T.ct}, {\tt T.pt}, {\tt T.cf}, {\tt T.pf} return a predicate\_table containing, respectively, the certainly true, possibly true, certainly false, and possibly false tuples in {\tt T}. % HIER ONTBREEKT __newindex
	\item[function\_interpretation] An interpretation for a function. {\tt F.graph} returns the predicate\_interpretation of the graph associated to the function\_interpretation {\tt F}. % HIER ONTBREEKT __newindex en __call
	\item[structure] A first-order structure. To obtain the interpretation of a sort, singleton predicate\_symbol, or singleton function\_symbol {\tt symb} in structure {\tt S}, write {\tt S[symb]}.
	\item[theory] A logic theory.
	\item[options] A set of options.
	\item[namespace] A namespace.
	\item[overloaded] An overloaded object.
\end{description}


%%%%%%%%%%%%%%%%%%%%%%%%%%%%%%%%%%%%%%%%%%%%%%%%%%%%%%%%
\subsection{Built-in procedures}
%%%%%%%%%%%%%%%%%%%%%%%%%%%%%%%%%%%%%%%%%%%%%%%%%%%%%%%%

A lot of procedures are already built-in. The command \code{help()} gives you an overview of all available sub-namespaces, procedures,\ldots. The \code{stdspace} namespace contains all built-in procedures.
\subsubsection{stdspace}
The \code{stdspace} contains the following procedures:
\begin{description}
\item[elements(d)]
	Returns a procedure, stopargument, and a beginindex (hence, a Lua-iterator) such that
	``for e in elements(d) do ... end'' 
	iterates over all elements in the given domain d.
\item[help(namespace)]
	List the procedures in the given namespace.
\item[idptype(something)]
	Returns custom typeids for first-class idp citizens.
\item[tuples(table)]
	Returns a Lua-iterator such that
	``for t in tuples(table) do ... end''
	iterates over all tuples in the given predicate table.
\end{description}
All procedures in \code{stdspace} are included in the global namespace and hence can be called by \code{procedure()} instead of by \code{stdspace.procedure()}.

Furthermore: \code{stdspace} contains the following subnamespaces:
\begin{description}
 \item[idpintern]
 \item[inferences]
 \item[options]
 \item[structure]
 \item[theory]
 \item[vocabulary]
\end{description}

\subsubsection{idpintern}
This namespace contains internal procedures of the \idp system. Using them is unsafe, since most procedures there are not tested well, or not documented. We recommend this only for advanced users. \code{idpinter} is not included in the global namespace; help for idpintern can be obtained by \code{help(stdspace.idpintern)}.

\subsubsection{inferences}
The \code{inferences} namespace and all its procedures are included in the global namespace. Hence \code{inferences.xxx} should never be used. This namespace contains several inference methods:
\begin{description}
	\item[calculatedefinitions(theory,structure)]
		Make the structure more precise than the given one by evaluating all definitions with known open symbols. This procedure works recursively: as long as some definition of which all open symbols are known exists, it calculates the definition (possibly deriving open symbols of other definitions).
		
	\item[ground(theory,structure)]
 		Create the reduced grounding of the given theory and structure. 
 	\item[groundeq(theory,structure,modeleq)]
 		Create the reduced grounding of the given theory and structure. modeleq is a boolean parameter:  whether or not the grounding should preserve the number of models (it always preserves satisfiability but might not preserve the number of models if modeleq is false).
 	\item[printgrounding(theory,structure)]
 		Print the reduced grounding of the given theory and structure.
 		MEMORY EFFICIENT: does not store the grounding internally.
 		
 	\item[groundpropagate(theory,structure)]
 		Return a structure, made more precise than the input by grounding and unit propagation on the theory.
		Returns nil when propagation makes the givens structure inconsistent.
	\item[optimalpropagate(theory,structure)]
 		Return a structure, made more precise than the input by generating all models and checking which literals always have the same truth value
 		This propagation is complete: everything that can be derived from the theory will be derived. 
 		Returns nil when propagation results in an inconsistent structure.
	\item[propagate(theory,structure)]
 		Returns a structure, made more precise than the input by doing symbolic propagation on the theory.
 		Returns nil when propagation results in an inconsistent structure.
 	
	\item[minimize(theory,structure,term)] Returns all models of the theory that extend the given structure and such that the term is minimal.
	
	\item[modelexpandpartial(theory,structure)]
		Apply model expansion to theory T, structure S.
 		The result is a table of (possibly three-valued) structures that are more precise then S and that satisfy T  and, if getoptions().trace == true, a trace of the solver.
		(this procedure is equivalent to first calling modelexpandpartial and subsequently calling alltwovaluedextensions. 
	\item[modelexpand(theory,structure)]
 		Apply model expansion to theory T, structure S.
 		The result is a table of two-valued structures that are more precise then S and that satisfy T and, if getoptions().trace == true, a trace of the solver.
		(this procedure is equivalent to first calling modelexpandpartial and subsequently calling alltwovaluedextensions. 
		
	\item[query(query,structure)]
 		Generate all solutions to the given query in the given structure. The result is the set of element-tuples that certainly satisfy the query in the structure.
	\item[sat(theory,structure)]
		Checks satisfiability of the given theory-structure combination. Returns true if and only if there exists a model extending the structure and satisfying the theory.

	\item[printcore(theory,structure)]
		Finds a theory that is a minimal ``subset'' of the given theory that is unsat in the given structure, or, in other words, a ``core''; it then prints this theory with 
		references to how it was obtained from the given theory.
		A subset in this respect is a subset of all instantiated formulas in conjunctive context in the theory and instantiated rules 
		(either all rules defining a specific domain atom/term or none of them).
		Currently does not take constraints implied by the vocabulary (e.g., function constraints) or the structure (e.g., type interpretations) into account.  
		
\end{description}

\subsubsection{options}
Like the \code{inferences} namespace, the \code{options} namespace and all its procedures are included in the global namespace. This namespace consists of the following procedures:

\begin{description}
	\item[getoptions()]
 		Get the current options.
	\item[newoptions()]
 		Create new options, equal to the standard options.
	\item[setascurrentoptions(options)]
 		Sets the given options as the current options, used by all other commands.
\end{description}

\subsubsection{structure}
Also this  namespace and all its procedures are included in the global namespace. Here, you can find several procedures for manipulating logical structures.

\begin{description}
	\item[alltwovaluedextensions(structure)] This procedures takes one (three-valued) structure and returns all structures over the same vocabulary that extend the given structure and are two-valued.
	\item[alltwovaluedextensions(table)]
		This procedures takes a table of structures and returns all two-valued extensions of any of the given structures.
	\item[clean(structure)]
		Modifies the given structure (the structure is the same, but its representation changes): transforms fully specified three-valued relations into two-valued ones.  For example if \code{P} has domain \code[1..2] and in the given structure, \code{P<ct> = \{1\}, P<cf> = \{2\}}, in the end, \code{P} is \code{\{1\}}.
	\item[clone(structure)]
		Returns a structure identical to the given one.
	\item[isconsistent(structure)]
 		Check whether the structure is consistent.
	\item[makefalse(predicate\_interpretation,table)]
 		Sets all tuples of the given table to false.
 		Modifies the table-interpretation.
	\item[makefalse(predicate\_interpretation,tuple)]
 		Sets the interpretation of the given tuple to false.
 		Modifies the table-interpretation.
	\item[maketrue(predicate\_interpretation,table)]
 		Sets all tuples of the given table to true.
 		Modifies the table-interpretation.
	\item[maketrue(predicate\_interpretation,tuple)]
 		Sets the interpretation of the given tuple to true.
 		Modifies the table-interpretation.
	\item[makeunknown(predicate\_interpretation,table)]
 		Sets all tuples of the given table to unknown.
 		Modifies the table-interpretation.
	\item[makeunknown(predicate\_interpretation,tuple)]
 		Sets the interpretation of the given tuple to unknown.
 		Modifies the table-interpretation.
	\item[newstructure(vocabulary,string)]
 		Create an empty structure with the given name over the given vocabulary.
 		
 	\item[createdummytuple()]
		Create an empty tuple.
	
	\item[iterator(domain)]
 		Create an iterator for the given sorttable.
	\item[iterator(predicate\_table)]
 		Create an iterator for the given predtable.
	\item[size(predicate\_table)]
 		Get the size of the given table.
 		
 	\item[range(number,number)]
 		Create a domain containing all integers between First and Last.
\end{description}


\subsubsection{theory}
The \code{theory} namespace and all its procedures are included in the global namespace. It contains methods for manipulating theories, most of which modify the given theory.
\begin{description}
	\item[clone(theory)]
		Returns a theory identical to the given one.
	\item[completion(theory)]
		Add definitional completion of all the definitions in the theory to the given theory. Modifies its argument.
	\item[flatten(theory)]
		Rewrites formulas with the same operations in their child formulas by reducing the nesting. For example $a \wedge (b\wedge c)$ will be rewritten to $a\wedge b \wedge c$.
		Modifies the given theory. 
	\item[merge(theory,theory)]
 		Create a new theory which is the result of combining (the conjunction of) both input theories.
	\item[pushnegations(theory)]
 		Push negations inwards until they are right in front of literals.
 		Modifies the given theory.
	\item[removenesting(theory)]
 		Move nested terms out of predicates (except for the equality-predicate) and functions.
 		Modifies the given theory.
\end{description}

\subsubsection{vocabulary}
The \code{vocabulary} namespace and all its procedures are included in the global namespace. It contains methods for setting and getting vocabularies.
\begin{description}
	\item[getvocabulary(query)]
		Returns the vocabulary of the given object.
	\item[getvocabulary(structure)]
		Returns the vocabulary of the given object.
	\item[getvocabulary(term)]
		Returns the vocabulary of the given object.
	\item[getvocabulary(theory)]
		Returns the vocabulary of the given object.
	\item[setvocabulary(structure,vocabulary)]
		Changes the vocabulary of a structure to the given one.
\end{description}










\subsubsection{Miscellaneous}
\todo{WHY is the parse method directly in global\_namespace?}
\begin{description}
	
	
	\item[parse(string)]
 		Parses the given file and adds its information into the datastructures.

\end{description}





\subsubsection{The mx library}
The mx standard library file contains some useful commands for model expansion. It can be used by 
\begin{lstlisting}
	#include <mx>
\end{lstlisting}
All commands in the mx standard library file are defined in a namespace with the same name (and hence should be called by \code{mx::command(...)}.
The following commands are supported by mx:
\begin{description}
	\item[one(theory,structure)]
		Does model expansion but only searches for one model (no matter what the \code{nbmodels} option is set to). Returns this structure (in contrast to the standard modelexpansion which returns a list of structures).
	\item[all(theory,structure)] 
		Returns all models of the theory that extend the given structure.
	\item[printmodels(list)]
		Prints a given list of models or prints unsatisfiable if the list is empty.
\end{description}

\subsubsection{The table\_utils library}
The table\_utils standard library file can be include by 
\begin{lstlisting}
	#include <table_utils>
\end{lstlisting}
It contains several useful commands for manipulating tables, converting predicate tables to lua-tables,\ldots.
\todo{document}

