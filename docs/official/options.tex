\section{Options}
\todo{How does the verbosity work?}
The \idp system has various options. To print the current values of all options, use \code{print(getoptions())}.  To set an option, you can use the following lua-code
\begin{lstlisting}
	stdoptions.MyOption = MyValue
\end{lstlisting}
where \code{MyOption} is the name of the option and \code{MyValue} is the value you want to give it. If you want to have multiple option sets, you can make them with them with
\begin{lstlisting}
	FirstOptionSet = newOptions()
	SecondOptionSet = newOptions()
	FirstOptionSet.MyOption = MyValue
	SecondOptionSet.MyOption = MyValue
\end{lstlisting}
To activate an option set, use the procedure \code{setascurrentoptions(MyOptionSet)}.  From that moment, \code{MyOptionSet} will be used in all comands.

\subsection{Verbosity options}
\begin{description}
	\item[{groundverbosity = [0..max(int)]}] Verbosity of the grounder.  The higher the verbosity, the more debug information is printed.
	\item[{satverbosity = [0..max(int)]}] Like groundverbosity, but controls the verbosity of \minisatid
	\item[{propagateverbosity = [0..max(int)]}] Like groundverbosity, but controls the verbosity of the propagation.
	\item[{calculatedefinitions = [0..5]}] Verbosity of the module that calculates definitions prior to grounding and solving. 
		\begin{itemize}
		\item 1 or greater: print when calculating the definition starts and ends.
		\item 2 or greater: print for each definition separately when it is being calculated and print when it is being calculated using \xsb.
		\item 3 or greater: print \xsb code that will be used to calculate the definitions.
		\item 5 or greater: print the structure that the calculatedefinitions operation returns.
		\end{itemize}
\end{description}

\subsection{Modelexpansion options}
\begin{description}
	\item[{nbmodels = [0..max(int)]}] Set the number of models wanted from the modelexpansion inference.  If set to 0, all models are returned.
	\item[{trace = [false, true]}] If true, the procedure modelexpand produces also an execution trace of \minisatid 
	%\item[{sharedtseitin = [false, true]}] Enable/disable a Tseitin transformation where subformulas are shared (hence some equivalent subformulas and certainly all syntactical equal subformulas have the same tseitin).
\end{description}

\subsection{Propagation options}
\begin{description}
	\item[{groundwithbounds = [false, true]}] Enable/disable bounded grounding (if enabled, first do symbolic propagation to provide ct and cf bounds for formulas to reduce the size of the grounding in every inferences that grounds (groundpropagate/ground/modelexpand/...)).
	\item[{longestbranch = [0..2147483647]}] The longest branch allowed in BDDs during propagation. The higher, the more precise the propagation will be (but also, the more time it will take).
	\item[{nrpropsteps = [0..2147483647]}] The number of propagation steps used in the propagate-inference. The higher, the more precise the propagation will be (but also, the more time it will take).
	\item[{relativepropsteps =  [false, true]}] If true, the total number of propagation steps is nrpropsteps multiplied by the number of formulas.

\end{description}

\subsection{Printing options}
\begin{description}
	\item[{language = [ecnf, idp, idp2 %latex, asp, txt
				tptp]}] The language used when printing objects. Note, not all languages support all kinds of objects.
	\item[{longnames = [false, true]}] If true, everything is printed with reference to their vocabulary.  For example, a predicate \code{P} from vocabulary \code{V} will be printed as \code{V::P} instead of \code{P}.
\end{description}

\subsection{General options}

\begin{description}
	%\item[{autocomplete = [false, true]}] Turn autocompletion of structures on or off
	%\item[{cpsupport = [false, true]}] Enable/disable cp-support (running a cp-solver together with the SAT-solver)
	%\item[{createtranslation = [false, true]}] \todo{watdoetdit?}
	%\item[{groundlazily = [false, true]}] Enable/disable lazy grounding
	%\item[{printtypes = [false, true]}] \todo{doet niets voor de moment?}
	%\item[{provertimeout = [0..2147483647]}] \todo{?}
	%\item[{showwarnings = [false, true]}] Enable/disable the warnings.
	%\item[{symmetry = [0..2147483647]}] \todo{?} 1 is static, 0 is none, higher = ?
	\item[{xsb = [false, true]}] Enable/disable the usage of \xsb
	\item[{timeout = [0..max(int)]}] Set the timeout for inferences (in seconds)
	\item[{seed = [0..max(int)]}] Set the seed for the random generator (used in the estimators for BDDs and in the SAT-solver)
	\item[{randomvaluechoice = [false, true]}] Controls the solver: if set to true, the assignment to choice literals is random, if set to false, the solver default assigns false to choice literals.
\end{description}
