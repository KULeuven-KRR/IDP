\section{Options}
\todo{How does the verbosity work?}
The \idp system has various options. To print the current values of all options, use \code{print(getoptions())}.  To set an option, you can use the following lua-code
\begin{lstlisting}
	stdoptions.MyOption = MyValue
\end{lstlisting}
where \code{MyOption} is the name of the option and \code{MyValue} is the value you want to give it. If you want to have multiple option sets, you can make them with them with
\begin{lstlisting}
	FirstOptionSet = newOptions()
	SecondOptionSet = newOptions()
	FirstOptionSet.MyOption = MyValue
	SecondOptionSet.MyOption = MyValue
\end{lstlisting}
To activate an option set, use the procedure \code{setascurrentoptions(MyOptionSet)}.  From that moment, \code{MyOptionSet} will be used in all comands.

\subsection{Verbosity options}
\begin{description}
	\item[{groundverbosity = [0..max(int)]}] Verbosity of the grounder.  The higher the verbosity, the more debug information is printed.
	\item[{satverbosity = [0..max(int)]}] Like groundverbosity, but controls the verbosity of \minisatid
	\item[{propagateverbosity = [0..max(int)]}] Like groundverbosity, but controls the verbosity of the propagation.
	\item[{approximatingdefinition = [0..5]}] Verbosity of the module that uses an approximating definition to perform approximation prior to grounding and solving. 
	\item[{calculatedefinitions = [0..5]}] Verbosity of the module that calculates definitions prior to grounding and solving. 
		\begin{itemize}
		\item 1 or greater: print when calculating the definition starts and ends.
		\item 2 or greater: print for each definition separately when it is being calculated and print when it is being calculated using \xsb.
		\item 3 or greater: print \xsb code that will be used to calculate the definitions.
		\item 4 or greater: print the structures that the calculatedefinitions operation uses and the structure that it returns.
		\item 5 or greater: print queries that are being sent to \xsb and the answer tuples they return.
		\end{itemize}
\end{description}

\subsection{Modelexpansion options}
\begin{description}
	\item[{nbmodels = [0..max(int)]}] Set the number of models wanted from the modelexpansion inference.  If set to 0, all models are returned.
	\item[{trace = [false, true]}] If true, the procedure modelexpand produces also an execution trace of \minisatid.
	\item[{cpsupport = [false, true]}] If true, constants can occur in the grounding, which are taken care of through integration with Constraint Programming techniques. 
	\item[{cpgroundatoms = [false, true]}] If true, the grounding can be full ground FO(.), which has an even smaller grounding (but might have reduced propagation).
	\item[{skolemize = [false, true]}] If true, existential quantification in sentences is replaced by introducing new function symbols. Only advantageous with cpsupport on.
	\item[{tseitindelay = [false, true]}] If true, grounding can be delayed by lazily expanding quantifications and disjunctions/conjunctions.
	\item[{satdelay = [false,true]}] If true, grounding can be delayed by maintaining justifications for non-ground sentences and rules.
	%\item[{sharedtseitin = [false, true]}] Enable/disable a Tseitin transformation where subformulas are shared (hence some equivalent subformulas and certainly all syntactical equal subformulas have the same tseitin).
\end{description}

\subsection{Propagation options}
\begin{description}
	\item[{groundwithbounds = [false, true]}] Enable/disable bounded grounding (if enabled, first do symbolic propagation to provide ct and cf bounds for formulas to reduce the size of the grounding in every inferences that grounds (groundpropagate/ground/modelexpand/...)).
	\item[{longestbranch = [0..2147483647]}] The longest branch allowed in BDDs during propagation. The higher, the more precise the propagation will be (but also, the more time it will take).
	\item[{nrpropsteps = [0..2147483647]}] The number of propagation steps used in the propagate-inference. The higher, the more precise the propagation will be (but also, the more time it will take).
	\item[{relativepropsteps =  [false, true]}] If true, the total number of propagation steps is nrpropsteps multiplied by the number of formulas.
\end{description}

\subsection{Printing options}
\begin{description}
	\item[{language = [ecnf, idp, idp2 %latex, asp, txt
				tptp]}] The language used when printing objects. Note, not all languages support all kinds of objects.
	\item[{longnames = [false, true]}] If true, everything is printed with reference to their vocabulary.  For example, a predicate \code{P} from vocabulary \code{V} will be printed as \code{V::P} instead of \code{P}.
\end{description}

\subsection{Entailment options}
\begin{description}
	\item[{provercommand = string}] String is the command by which a theorem prover can be called (as on a command-line). It has to contain the placeholders \code{\%i} and \code{\%o} which will be replaced with the input and output file, respectively.
	\item[{proversupportsTFA = [false, true]}] Should be set to true if the selected prover (using above command) supports to TFA syntax of the CASC competition (Typed Fo with Arithmetic). Otherwise FOF syntax (First-Order Formulas) will be used.
\end{description}

\subsection{General options}

\begin{description}
	%\item[{autocomplete = [false, true]}] Turn autocompletion of structures on or off
	%\item[{cpsupport = [false, true]}] Enable/disable cp-support (running a cp-solver together with the SAT-solver)
	%\item[{createtranslation = [false, true]}] \todo{watdoetdit?}
	%\item[{groundlazily = [false, true]}] Enable/disable lazy grounding
	%\item[{printtypes = [false, true]}] \todo{doet niets voor de moment?}
	%\item[{provertimeout = [0..2147483647]}] \todo{?}
	%\item[{showwarnings = [false, true]}] Enable/disable the warnings.
	%\item[{symmetry = [0..2147483647]}] \todo{?} 1 is static, 0 is none, higher = ?
	\item[{assumeconsistentinput = [false, true]}] If true, input structures are not checked for consistency (which can be expensive to verify). 
	\item[{xsb = [false, true]}] Enable/disable the usage of \xsb.
	\item[{timeout = [0..max(int)]}] Set the time after which an inference is requested to stop gracefully (in seconds, 0 indicates \emph{no} timeout).
		In interactive mode, this is one lua call, otherwise the execution of the command supplied by ``-e'' (or the main procedure) is timed.
		Whenever a timeout is reached, the inference is provided with 10 seconds to exit gracefully, otherwise the system aborts.				
	\item[{mxtimeout = [0..max(int)]}] Set the time after which an execution of model expansion times out (in seconds, 0 indicates \emph{no} timeout).
		If any models have been found, they are returned properly.
		In case of model optimization, the best model(s) found to date are returned, not guaranteeing optimality has been proven. 
	\item[{seed = [0..max(int)]}] Set the seed for the random generator (used in the estimators for BDDs and in the SAT-solver).
	\item[{approxdef = ["none", "all", "cheap", "stratified"]}]
		\begin{description}
		\item [none]: do not use any approximating definition
		\item [all]: use the complete approximating definition
		\item [cheap]: use the approximating definition without certain expensive rules
		\item [stratified]: first calculate the cheap rules, then calculate the non-cheap
		\end{itemize}
	\item[{randomvaluechoice = [false, true]}] Controls the solver: if set to true, the assignment to choice literals is random, if set to false, the solver default assigns false to choice literals.
\end{description}
