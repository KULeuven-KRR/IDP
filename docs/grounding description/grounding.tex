\documentclass{article}

\usepackage{xspace}
\usepackage{listings}
\usepackage{url}
\usepackage[usenames,dvipsnames]{color}
\usepackage{amssymb, amsmath}

\newcommand{\code}[1]{{\tt #1}}
\newcommand{\todo}[1]{\textcolor{blue}{TODO: #1}}
\newcommand{\ignore}[1]{}

\renewcommand{\bold}[1]{\textbf{#1}}

%% Logics taxonomy
\newcommand{\idp}{{\sc IDP}\xspace}
\newcommand{\idptwo}{{\sc IDP}$^2$\xspace}
\newcommand{\idpthree}{{\sc IDP}$^3$\xspace}
\newcommand{\minisatid}{{\sc MiniSAT(ID)}\xspace}
\newcommand{\gidl}{{\sc GidL}\xspace}
\newcommand{\fodotidp}{{\sc FO($\cdot$)$^{\mathtt{\tiny IDP}}$}\xspace}
\newcommand{\foidp}{\fodotidp}
\newcommand{\fodot}{{\sc FO($\cdot$)}\xspace}

%% Logic
% lor, land, lnot
\newcommand{\bigand}{\ensuremath{\bigwedge}}
\newcommand{\bigor}{\ensuremath{\bigor}}
\newcommand{\limplies}{\ensuremath{\Rightarrow}}
\newcommand{\limpliedby}{\ensuremath{\Leftarrow}}
\newcommand{\lequiv}{\ensuremath{\Leftrightarrow}}
\newcommand{\lrule}{\ensuremath{\leftarrow}}

\newcommand{\voc}{\ensuremath{\Sigma}}
\newcommand{\invoc}{\ensuremath{\sigma_{in}}}
\newcommand{\outvoc}{\ensuremath{\sigma_{out}}}
\newcommand{\instruct}{\ensuremath{J_{in}}}
\newcommand{\outstruct}{\ensuremath{J_{out}}}
\newcommand{\theory}{\ensuremath{T}\xspace}

%% TWO-VALUED
\newcommand{\true}{\ensuremath{\top}\xspace}
\newcommand{\false}{\ensuremath{\bot}\xspace}

%% THREE-VALUED
\newcommand{\ltrue}{\textbf{t}}
\newcommand{\lfalse}{\textbf{f}}
\newcommand{\lunkn}{\textbf{u}}

\newcommand{\atom}{\ensuremath{a}\xspace}
\newcommand{\lit}{\ensuremath{l}\xspace}

\newcommand{\rules}{\ensuremath{R}\xspace}
\newcommand{\defined}[1]{\ensuremath{#1_{def}}\xspace}
\newcommand{\open}[1]{\ensuremath{#1_{open}}\xspace}

\newcommand{\xxx}{\ensuremath{\overline{x}}}
\newcommand{\yyy}{\ensuremath{\overline{y}}}
\newcommand{\zzz}{\ensuremath{\overline{z}}}
\newcommand{\ddd}{\ensuremath{\overline{d}}}
\newcommand{\ttt}{\ensuremath{\overline{t}}}

\renewcommand{\|}{\ensuremath{\,|\,}}
\newcommand{\elim}{\ensuremath{\backslash}}

\newcommand{\subs}{\ensuremath{/}}

\newcommand{\func}[1]{\ensuremath{f(#1)}\xspace}

\newcommand{\set}{\ensuremath{S}\xspace}
\newcommand{\inter}{\ensuremath{I}\xspace}
\newcommand{\partinter}{\ensuremath{J}\xspace}

\newcommand{\fone}{\ensuremath{\varphi}\xspace}
\newcommand{\ftwo}{\ensuremath{\psi}\xspace}


%% Inferences
\newcommand{\mx}[3]{\ensuremath{<#1, #2, #3>}}

%% Source code fragments
\usepackage{listings}

\lstdefinelanguage{idp}{
	morekeywords=[1]{namespace,vocabulary,theory,structure,procedure},
	morekeywords=[2]{include,using,type,isa,contains,partial,extern,LFD,GFD},
	morekeywords=[3]{int,float,char,string,nat},
	morecomment=[s]{/*}{*/},	
	morecomment=[l]{//}
}
\lstset{
	language=idp,
	tabsize=3,
	frame=none,
	basicstyle=\ttfamily,
	commentstyle=\color{Gray},
	keywordstyle=[1]\color{BrickRed}\bfseries,
	keywordstyle=[2]\color{OliveGreen}\bfseries,
	keywordstyle=[3]\color{Blue}\bfseries,
}


% \newcommand{\voc}{\Sigma}
\newcommand{\I}{\mathcal{I}}
\newcommand{\C}{\mathcal{C}}
\newcommand{\A}{\mathcal{A}}
\newcommand{\seq}[1]{\left\langle#1\right\rangle}

\newcommand{\vocc}{\voc'}
\newcommand{\T}{\mathcal{T}}
\newcommand{\Tt}{\mathcal{T}_t}
\newcommand{\Ts}{\mathcal{T}_s}
\newcommand{\Tg}{\mathcal{T}_g}
\newcommand{\Td}{\mathcal{T}_d}
\newcommand{\Tdef}{\mathcal{T}_\Delta}
\newcommand{\Tr}{\mathcal{T}_r}

\newcommand{\pg}{\seq{\voc,\T}}
\newcommand{\pgg}{\seq{\voc',\T'}}

\newcommand{\rewrite}{\mapsto}

\newcommand{\indot}{{\sc FO($\cdot$)$_{int}$}\xspace}
\newcommand{\DDD}{\ensuremath{\overline{D}}}

% not for llncs
\newtheorem{definition}{Definition}
\newtheorem{example}{Example}
%\newtheorem{proof}{Proof}
\newtheorem{lemma}{Lemma}
\newtheorem{proposition}{Proposition}
\newtheorem{theorem}{Theorem}
\newtheorem{corollary}{Corollary}


\title{Partial grounding}

\date{}

\begin{document}

\maketitle

Let $T$ be a theory over $\voc$ and $\I$ a partial $\voc$-interpretation.
Let $T_{Dom}$ be the bounded formula obtained by replacing each $\forall x$ by $\forall x[Dom]$ and $\exists x$ by $\exists x[Dom]$ in $T$.

\begin{definition}[Delayed sentence, delayed theory]
	A delayed theory is a set of delayed sentences: triples $\left\langle \phi, \delta, c \right\rangle$, where $\phi$ is a formula, $\delta$ is a query and $c$ is a model construction process, i.e. a mapping from interpretations to interpretations.
\end{definition}

\todo{De onderstaande definitie is nog iets te sterk: zou enkel moeten voldaan zijn voor alle uitbreidingen van de structuur waarme we beginnen}
\begin{definition}[Allowed]
	A delayed sentence is said to be \textit{allowed} if for every partial interpretation $I$ such that $\delta^I = \emptyset$: $c(\I)$ is consistent and $\varphi^{c(\I)}$ is true, i.e.: $c$ completes $\I$ to a partial interpretation satisfying $\varphi$.

	A delayed theory is \textit{allowed} if there exists a series $i_1, i_2,\cdots$ such that the joint construction process ($c_{i_1} \circ c_{i_2} \circ \cdots$) completes any partial interpretations $\I$ for which all queries are empty to an interpretation in which every $\varphi$ evaluates to true.
\end{definition}

\begin{definition}[Component]
A \textit{component} is either
\begin{itemize}
\item a definition,
\item a sentence,
\item or a delayed sentence.
\end{itemize}
\end{definition}

\begin{definition}[Partial grounding]
	A partial grounding  is a tuple $\pg$, where $\voc$ is a vocabulary and $\T$ is a set of components over $\voc$. We often identify the partial grounding with its set of components. If $\T$ is a partial grounding, we denote by $\Ts$ the set of its non-delayed sentences, by $\Tg$ the set of its ground sentences (which is thus a subset of $\Tt$). We also use $\Td$ for the set of delayed sentences and $\Tdef$ for the set of definitions. 

With every partial grounding, we associate an \fodot-theory $\Tt$, obtained by replacing every delayed sentence by its underlying sentence.
\end{definition}

We only consider allowed partial groundings, i.e. partial groundings $\pg$ where $\Td$ is allowed.

\begin{definition}[Semantics of a partial grounding]
	A $\voc$-structure $\I$ is said to satisfy partial grounding $\pg$ iff it satisfies $\Tt$.
\end{definition}

\begin{definition}[Specialization]
	A partial grounding $\pgg$ is said to be specialization of $\pg$ with respect to $\I$ if the following hold:
\begin{itemize}
	\item $\vocc$ is an extension of $\voc$
	\item The mapping that restricts a $\vocc$-structure to a $\voc$-structure induces a one-to-one correspondence between models of $\T'$ extending $\I$ and models of $\T$ extending $\I$.
\end{itemize}
\end{definition}

\begin{definition}[Partial grounding of $T$ and $\I$]
	A partial grounding $\pg$ is said to be a partial grounding of $T$ and $\I$ if it is a specialization of $\seq{\voc,T}$ with respect to $\I$.
\end{definition}



\begin{definition}[Condition]
A condition on a formula $\varphi$ in $\pg$ is an expression of one of the following types:
\begin{itemize}
	\item A description of the nature of $\varphi$, for example ``$\varphi = \forall x[D]: \psi$''.
	\item A description of the context of $\varphi$, for example, ``$\varphi$ occurs monotone in the body of a rule'' or ``$\varphi$ is a subformula of a delayed sentence''.
	\item ...
\end{itemize}
\end{definition}

\begin{definition}[Action]
	If $\varphi_i$ are formulas occurring somewhere in $\pg$, an action on $\pg$ is one of the following expressions:
\begin{itemize}
	\item $\T += \{\psi\}$, (this means: ``add sentence $\psi$ to theory $\Tt$'')
	\item $\voc +=\{P[T_1,\cdots,T_n]\}$ (means ``add a new symbol $P$ to the vocabulary'')
	\item $\varphi \mapsto \psi$ (this means: ``replace $\varphi$ by $\psi$'')
    \item ...
\end{itemize}

\end{definition}




\begin{definition}[(Rewrite rule]
A rewrite rule $r$ is a set of conditions $\C_r$ together with a set of actions $\A_r$.

The result of applying rule $r$ to partial grounding $\pg$ is denoted $r(\pg)$ . Rewrite rule $r$ is said to be allowed if for every partial grounding $\pg$, $r(\pg)$ is a specialization of $\pg$.	 
\end{definition}

\begin{example}[Tseitin transformation]
Conditions:
\begin{itemize}
	\item $\varphi$ is a subformula of a sentence in $\T$
	\item $\varphi$ is not an atom
	\item The free variables of $\varphi$ are $x_1,\cdots,x_n$ with types $T_1,\cdots,T_n$.
\end{itemize}
Actions:
\begin{itemize}
	\item $\voc +=\{Tseitin_i[T_1,\cdots,T_n]\}$, where $i$ is a new identifier.
	\item $\varphi \mapsto Tseitin_i(x_1,\cdots,x_n)$
	\item $\T += \{\forall x_1\cdots x_n: Tseitin_i\Leftrightarrow \varphi\}$.
\end{itemize}
\end{example}

\begin{example}[Shared Tseitin transformation]
Conditions:
\begin{itemize}
	\item $\varphi_1,\cdots,\varphi_k$ are equivalent subformulas of sentences in $\T$
    \item none of the $\varphi_i$ is an atom
    \item The free variables of $\varphi_i$ are $x_{i_1},\cdots,x_{i_n}$ with types $T_1,\cdots,T_n$.
\end{itemize}
Actions:
\begin{itemize}
	\item $\voc +=\{Tseitin_j[T_1,\cdots,T_n]\}$, where $j$ is a new identifier.
	\item for every i, $\varphi_i \mapsto Tseitin_j(x_{i_1},\cdots,x_{i_n})$
	\item $\T += \{\forall x_{i_1}\cdots x_{i_n}: Tseitin_i\Leftrightarrow \varphi_i\}_i$.
\end{itemize}
\end{example}

\begin{example}[Push negations inwards]
Conditions:
\begin{itemize}
	\item $\varphi = \lnot\left( \varphi_1 \lor \varphi_2\right)$
    \item or $\varphi = \lnot\left( \varphi_1 \land \varphi_2\right)$    
    \item or $\varphi = \lnot\forall x: \varphi_1[x]$    
    \item or $\varphi = \lnot\exists x: \varphi_1[x]$
\end{itemize}
Actions (in the respective cases):
\begin{itemize}
	\item $\varphi \mapsto\left(  \lnot \varphi_1 \land  \lnot \varphi_2\right)$
    \item or $\varphi \mapsto \left( \lnot \varphi_1 \lor \lnot \varphi_2\right)$
    \item or $\varphi \mapsto \exists x: \lnot\varphi_1[x]$
    \item or $\varphi \mapsto \forall x: \lnot\varphi_1[x]$ 
\end{itemize}
\end{example}

\begin{example}[Instantiation]
Conditions:
\begin{itemize}
    \item  $\varphi = \forall x[D]: \varphi_1[x]$
    \item or $\varphi = \exists x[D]: \varphi_1[x]$
\end{itemize}
Actions (in the respective cases):
\begin{itemize}
    \item  $\varphi \mapsto \varphi_1[d/x] \land  \forall x[D\setminus d]: \varphi_1[x]$
    \item or $\varphi \mapsto \varphi_1[d/x] \lor  \exists x[D\setminus d]: \varphi_1[x]$
\end{itemize}
\end{example}

\begin{example}[Positive Unnesting]
Conditions:
\begin{itemize}
    \item  $\varphi = P(\bar x, f(\bar y),\bar z)$, for some (partial) function $f$ and predicate/function $P$.
	\item $\varphi$ occurs in the scope of an even number of negations
\end{itemize}
Actions:
\begin{itemize}
    \item  $\varphi \mapsto \forall w: f(\bar y) = w \Rightarrow P(\bar x, f(\bar y),\bar z)$
\end{itemize}
\end{example}

\begin{example}[Negative Unnesting]
Conditions:
\begin{itemize}
    \item  $\varphi = P(\bar x, f(\bar y),\bar z)$, for some (partial) function $f$ and predicate/function $P$.
	\item $\varphi$ occurs in the scope of an odd number of negations
\end{itemize}
Actions:
\begin{itemize}
    \item  $\varphi \mapsto \exists w: f(\bar y) = w \wedge P(\bar x, f(\bar y),\bar z)$
\end{itemize}
\end{example}

\begin{example}[Functional Unnesting]
	Same as unnesting but with $f(x)=g(y)$.
\end{example}

\begin{example}[Graphing]
Conditions:
\begin{itemize}
    \item  All occurences of a function symbol $F$ are of the form $\varphi = F(\bar x) = y$, where $y$ is a domain element or a variable. (alternatively, we could omit the condition but force this in the partial grounding process using the partial ordering of actions)
\end{itemize}
Actions:
\begin{itemize}
    \item  $\varphi \mapsto GRAPH_F(\bar x, y)$ for all occurences $\varphi$ as described above
\item $\T += \{\forall \bar x: \#\{y: GRAPH_F(\bar x, y)\} = 1\}$.
\end{itemize}
\end{example}

\begin{definition}[Grounding description]
A grounding description is a set of allowed rewrite rules with a partial ordering.
\end{definition}

\begin{definition}[Partial grounding process]
A \textit{partial grounding process} of $T$ and $\I$ with description $D$ is a sequence $\seq{T_i}$ of partial groundings where $T_0= \seq{\voc,T}$ and such that for every $i>0$ there is a rule $r$ in $D$ such that:
\begin{itemize}
	\item $T_i = r(T_{i-1})$.
	\item The conditions of $r$ are satisfied.
	\item For every smaller rule $r'$, its conditions are  not satisfied
\end{itemize}
\end{definition}



\begin{definition}[ECNF grounding process]
	An ECNF grounding process is a grounding process of length $n$ such that $T_n=\pg$, where $\T = \Tg$ is in ECNF format.
\end{definition}


\begin{definition}[ECNF-CP grounding process]
	An ECNF-CP grounding process is a grounding process of length $n$ such that $T_n=\pg$, where $\Tg$ is in ECNF format and $\T\setminus \Tg$ consists of CP-constraints (todo definition).
\end{definition}

\begin{definition}[Delayed grounding process]
	A delayed grounding process is a grounding process of length $n$ such that $T_n=\pg$, where $\Tg$ is in ECNF format and $\T = \Tg \bigcup \Td$.
\end{definition}
\todo{Opmerking: op deze manier kunnen we dus onze verschillende soorten grounding processen specifieren.}


\section{Effective grounding process}
The grounding proceeds in two phases.
In phase one, the \fodot theory is rewritten into an intermediary format \indot, which can be grounded efficiently with a small number of rewrite rules. 
In the second phase, this intermediary format is grounded into \pcdot.

\subsection{Phase II}
Here we present the intermediary format \indot and the final ground format \pcdot and the rewrite rules and control used to convert the first into the second.

\section{\indot}
Intermediate format \indot reflects quite closely the grounder hierarchy that will be constructed for a given theory.

Formula, term, etc. are defined recursively as follows:
A basic term is a variable or a domain element.
A function term is a function over integers applied to variables and domain elements or addition or subtraction applied to function terms.
An aggregate term is an aggregate function applied to a set $\{t(\xxx) | \xxx \in \DDD, \psi\}$, with $\xxx$ variables, $t$ a term, $\DDD$ a domain vector and $\psi$ a formula. 
Note: both $t$, $\DDD$ and $\psi$ can depend on additional free variables.
A term is a basic, function or aggregate term.

A formula is 
\begin{itemize}
  \item an atom $P(\ttt)$, with $\ttt$ basic terms.
  \item an atom $t \sim t'$, with $t$ a term and $t'$ a non-aggregate term, $\sim$ is one of $\{=,~\geq,~\leq,~>,~<,~\neq\}$.
  \item applying $\land,~\lor,~\lnot$ to formulas.
  \item $\forall \xxx[\DDD]: \psi$($\exists \xxx[\DDD]: \psi$), with $\xxx$ variables, $\DDD$ a domain vector and $\psi$ a formula.
  \item Nothing else is a formula. 
\end{itemize}

For each symbol, we have an interpreted sort or all its arguments.

\inter represents a partial interpretation.

A rule is an expression of the form $\forall \xxx[\DDD]: P(\xxx) \lrule \psi$, with $\psi$ a formula and $P(\xxx)$ an atom. 
TODO type of $P$.
A definition is a set of rules.

TODO domain is in fact a query over a logical formula with only known symbols


\section{\pcdot}
The ground format, which we want to obtain as the limit of the rewriting procedure.

Upper-case characters are propositional variables, lower-case characters are finite domain integer variables.
If they are preceded with a $\_$, their value is known.

An expression is one of the following:
\begin{itemize}
  \item Clauses $P \lor Q \lor \ldots$.
  \item $P \sim\sim Q_1 \lor \ldots \lor Q_n$. $\sim\sim$ stands for $\limplies$, $\limpliedby$, $\equiv$ or $\lrule$.
  \item $P \sim\sim Q_1 \land \ldots \land Q_n$.
  \item Disjunctive rule $P \lrule Q_1 \lor \ldots \lor Q_n$.
  \item Conjunctive rule $P \lrule Q_1 \land \ldots \land Q_n$.
  \item An fdvariable declaration $v \in [\_i, \_j]$.
  \item A comparison $P \equiv v_1 \sim v_2$.
  \item A comparison $P \equiv v \sim \_i$.
  \item An aggregate expression $P \sim\sim agg(\{\_i_1 \times P_1 \times v_1,\ldots,\_i_n \times P_n \times v_n\}) \sim \_i$. 
  \item FUTURE: An function application $P \equiv Q(\ttt)$, with $\ttt$ a list of domain elements and fdvariables. If $Q$ is equality, on argument is allowed to be a function term applied to domain elements and fdvariables.
\end{itemize}

TODO what is in fact the output vocabulary?

TODO rules will be replaced with an unfounded set constraint and explicit completion.

TODO generalize this format to something consistent and consider the internal rewritings into this format as really internal.

\section{Rewriting \indot to \pcdot}
This algorithm lies the closest to what is really implemented, but might not be how we want to represent it.

The algorithm maintains some state:
\begin{itemize}
  \item Tseitin interpretations $Ts$: a mapping of propositional atoms to their formula interpretation. These interpretations are always formulas that can be directly added to the grounding.
	\item Grounding $G$ in \pcdot.
	\item Remaining theory $R$ in \indot.
\end{itemize}

\subsection{Rewriting rules}
An uppercase character always represents an atom with only domain elements as arguments.

Assume we have a domain element $d_{NULL}$, for which the type check always fails. Also, one literal exists which represents \true (its negation then represents \false).

Instantiating a variable with a domain element in a formula replaces all occurrences of the variable, also in domains. 

TODO partial functions

\begin{itemize}
  \item Type checks
  \item Formula to tseitin
  \item Simplifying conjunction, disjunction
  \item Instantiation
  \item Add to grounding
  \item Set to ground set
  \item Rule to propositional rule + false defineds
  \item Optimization
  \item Lazy stuff
\end{itemize}

\noindent Type checks

\begin{tabular}{|lcl|} \hline
$P(t_1, \ldots, t_n)$ in $R$ & $\Longrightarrow$& $P(t_1, \ldots, t_n)\mapsto \false$\\
$t_i$ is a domain element && \\
$t_i \not\in Sort(P,i)$ &&\\
\hline
\end{tabular}

\begin{tabular}{|lcl|} \hline
$f(t_1, \ldots, t_n)$ in $R$ & $\Longrightarrow$& $f(t_1, \ldots, t_n)\mapsto d_{NULL}$\\
$t_i$ is a domain element && \\
$t_i \not\in Sort(f,i)$ &&\\
\hline
\end{tabular}

\noindent Tseitin introduction

\begin{tabular}{|lcl|} \hline
$\varphi = Q_1 \land \ldots \land Q_n$ & $\Longrightarrow$& $\varphi\mapsto T$\\
$\psi = P_1 \lor \varphi \lor \ldots \lor P_n $ && $Ts += (T \sim\sim(\text{context}) \varphi)$\\
$\psi \in R$, context &&\\
\hline
\end{tabular}

\begin{tabular}{|lcl|} \hline
$\varphi = t \sim t'$ in $R$ & $\Longrightarrow$& $\varphi\mapsto T$\\
$t$ is a constant over int && $Ts += (T \equiv t \sim t'$\\
$t'$ is a domain element or constant over int&&\\
\hline
\end{tabular}

TODO allow swapping order

\begin{tabular}{|lcl|} \hline
$\varphi = agg(\{\_i_j,P_j,t_j\}) \sim t'$ in $R$ & $\Longrightarrow$& $\varphi\mapsto T$\\
$t'$ is an integer && $Ts += (T \sim\sim(\text{context}) agg(\{\_i_j,P_j,t_j\}) \sim t'$\\
\hline
\end{tabular}

\noindent Disjunction reduction

\begin{tabular}{|lcl|} \hline
$\varphi=\psi \lor \true$ & $\Longrightarrow$ & $\varphi \mapsto \true$ \\
$\varphi \in R$ &&\\
\hline
\end{tabular}

\begin{tabular}{|lcl|} \hline
$\varphi=\psi \lor \false$ & $\Longrightarrow$ & $\varphi \mapsto \psi$ \\
$\varphi \in R$ &&\\
\hline
\end{tabular}

\noindent Conjunction reduction

\begin{tabular}{|lcl|} \hline
$\varphi=\psi \land \true$ & $\Longrightarrow$ & $\varphi \mapsto \psi$ \\
$\varphi \in R$ &&\\
\hline
\end{tabular}

\begin{tabular}{|lcl|} \hline
$\varphi=\psi \land \false$ & $\Longrightarrow$ & $\varphi \mapsto \false$ \\
$\varphi \in R$ &&\\
\hline
\end{tabular}

\noindent Instantiation

\begin{tabular}{|lcl|} \hline
$\varphi = \forall \xxx[\DDD]: \psi$ in $R$ & $\Longrightarrow$ & $\varphi \mapsto \psi[\xxx\subs\ddd] \land \forall \xxx[\DDD\elim\ddd]: \psi$\\
$\ddd \in \DDD$ && \\
\hline
\end{tabular}

\begin{tabular}{|lcl|} \hline
$\varphi = \exists \xxx[\DDD]: \psi$ in $R$ & $\Longrightarrow$ & $\varphi \mapsto \psi[\xxx\subs\ddd] \lor \exists \xxx[\DDD\elim\ddd]: \psi$\\
$\ddd \in \DDD$ && \\
\hline
\end{tabular}

\noindent Add to grounding

\begin{tabular}{|lcl|} \hline
$\varphi = P_1 \lor \ldots \lor P_n$ & $\Longrightarrow$ & $R-=\{\varphi\}$ \\
$\varphi$ is a sentence $R$ && $G+=\{\varphi\}$\\
\hline
\end{tabular}

\begin{tabular}{|lcl|} \hline
$P$ atom in $G$ & $\Longrightarrow$ & $G+=\{P \sim\sim \varphi\}$ \\
$P \sim\sim \varphi$ in $Ts$ && $Ts -= (T \sim\sim \varphi)$\\
\hline
\end{tabular}

\noindent Set to ground set

\begin{tabular}{|lcl|} \hline
$S=\{t | \xxx \in \DDD, \psi\}$ in $R$ & $\Longrightarrow$ & $S \mapsto \{1\times \psi[\xxx\subs\ddd] \times t[\xxx\subs\ddd]\}\cup\{t | \xxx \in \DDD\elim\ddd, \psi\}$ \\
$\ddd \in \DDD$ &&\\
\hline
\end{tabular}

\begin{tabular}{|lcl|} \hline
$S=\{1\times \psi \times t\}$ in $R$ & $\Longrightarrow$ & $S \mapsto \{1\times T \times t\}$ \\
$\psi$ is not an atom & & $T += \{ T \sim\sim \psi\}$ \\
\hline
\end{tabular}

Unions of ground sets can be directly combined.
The weight might be optimized to be non-zero when $t$ is in fact $constant \times variable$ (is this useful anyway?).

\noindent Rules

\begin{tabular}{|lcl|} \hline
$\varphi = P \lrule Q_1 \lor \ldots \lor Q_n$ & $\Longrightarrow$ & $\Delta_i-=\{\varphi\}$ \\
$\varphi$ is a rule in definition $\Delta_i$ in $R$ && $G+=\{P \lrule_i Q_1 \lor \ldots \lor Q_n\}$\\
\hline
\end{tabular}

\begin{tabular}{|lcl|} \hline
$P$ is a symbol in $R$ & $\Longrightarrow$ & $G+=\{P \lrule \false\}$ \\
$P$ is defined in some definition in $R$ or $G$ &&\\
There is no rule which can derive $P(\ddd)$, $\ddd \in$ domain $P$ &&\\
\hline
\end{tabular}


\noindent CP

\section{Propagation over the new function constraints}
Equality solver to derive more equalities, resulting in more propagation.
\end{document}
