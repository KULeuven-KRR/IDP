\chapter{{\tt perlmatch}: Perl-based String Manipulation}

\begin{center}
{\Large {\bf By Michael Kifer and  Jin Yu }}
\end{center}

XSB has an efficient interface to the Perl interpreter, which allows
XSB programs to use powerful Perl pattern matching capabilities. this
interface is provided by the {\tt Perlmatch} package. You need Perl
5.004 or later to be able to take advantage of this
service~\footnote{This packages is not yet supported by the
  multi-threaded engine.}.

This package is mostly superseded by the the more efficient POSIX {\tt
  Regmatch} package described in the previous section. However, Perl
regular expressions provide certain features not available in the {\tt
  Regmatch} package, such as the ability to perform global
replacements of matched substrings. Also, the {\tt Perlmatch} package
has a different programming interface, which is modeled after the
interface provided by Perl itself. So, if you are a big fan of Perl,
this package is for you.

The following discussion assumes that you are familiar with the syntax of
Perl regular expressions and have a reasonably good idea about the
capabilities of Perl matching and string substitution functions.

In the interactive mode, you must first load the {\tt Perlmatch}  package:
%%
\begin{quote}
 {\tt  :- [perlmatch]. }
\end{quote}
%%

In a program, you must import the package predicates:
%%
\begin{verbatim}
:- import bulk_match/3, get_match_result/2, try_match/2,
	  next_match/0, perl_substitute/3, load_perl/0, unload_perl/0
   from perlmatch.
\end{verbatim}
%%

\section{Iterative Pattern Matching}
There are two ways to do matching. One is to first do the matching
operation and then count the beans. To find the first match, do:

%%
\begin{quote}
 {\tt   :- try\_match( +String, +Pattern ). }
\end{quote}
%%

Both arguments must be of the XSB string data types.  If there is a match
in the string, the submatches {\$}1, {\$}2, etc., the prematch substring
{\$}` ({\it i.e.}, the part before the match), the postmatch substring
{\$}' ({\it i.e.}, the part after the match), the whole matched substring
{\$}{\&}, and the last parentheses match substring {\$}+ will be stored
into a global data structure, and the predicate try\_match(string, pattern)
succeeds. If no match is found, this predicate fails.

The ability to return parts of the match using the Perl variables {\$}+,
{\$}1, {\$}2, etc., is an extremely powerful feature of Perl.
As we said, a familiarity with Perl matching is assumed, but we'll give an
example to stimulate the interest in learning these capabilities of Perl.
For instance, \verb|m/(\d+)\.?(\d*)|\.(\d+)/| matches a valid
number. Moreover, if the number had the form `xx.yy', then the Perl variable
\$1 will hold `xx' and \$1 will hold `yy'. If the number was of the form
`.zz', then \$1 and \$2 will be empty, and \$3 will hold `zz'.

XSB-Perl interface provides access to all these special variables using the
predicate {\tt get\_match\_result()}.
The input variables string
and pattern are of XSB string data types. For example:

For instance,
%%
\begin{quote}
 {\tt :- try\_match('First 11.22 ; next: 3.4', 'm/(\d+)\.?(\d*)/g'). }\\
 {\tt yes.}  
\end{quote}
%%
finds the character which precedes by 's' in the string 'this is a
test'. The first match is `is'.

Now, we can use {\tt get\_match\_result()} to find the submatches.  The
first argument is a tag, which can be 1 through 20, denoting the Perl
variables \$1 -- \$20. In addition, the entire match can be found using the
tag {\tt match}, the part before that is hanging off the tag {\tt prematch}
and the part of the string to the right of the match is fetched with the
tag {\tt postmatch}. For instance, in the above, we shall have: 

This function is used to fetch the pattern match results {\$}1, {\$}2, etc.,
{\$}`, {\$}', {\$}{\&} and {\$}+, as follows:

\begin{verbatim}
:- get_match_result(1,X).
X=11
yes
:- get_match_result(2,X).
X=22
yes
:- get_match_result(3,X).
no
:- get_match_result(4,X).
no
:- get_match_result(prematch,X)
X=First  (including 1 trailing space)
yes
:- get_match_result(postmatch,X)
X= ; next 3.4
yes
:- get_match_result(match,X)
11.22
yes
\end{verbatim}

As you noticed, if a tag fetches nothing (like in the case of Perl
variables \$3, \$4, etc.), then the predicate fails.

The above is not the only possible match, of course. If we want more, we
can call:
%%
\begin{quote}
 {\tt  :- next\_match. }
\end{quote}
%%

This will match the second number in the string. Correspondingly, we shall
have: 
\begin{verbatim}
   :- get_match_result(1,X).
   X=3
   yes
   :- get_match_result(2,X).
   X=4
   yes
   :- get_match_result(3,X).
   no
   :- get_match_result(4,X).
   no
   :- get_match_result(prematch,X)
   X=First 11.22 ; next 
   yes
   :- get_match_result(postmatch,X)
   no
   :- get_match_result(match,X)
   3.4
   yes
\end{verbatim}

The next call to {\tt next\_match}  would fail, because there are no more
matches in the given string. Note that {\tt next\_match} and
{\tt get\_match\_result} do not take a string and a pattern as
argument---they work off of the previous {\tt try\_match}. If you want to
change the string or the pattern, call {\tt try\_match} again (with
different parameters).

\paragraph{Note:} To be able to iterate using {\tt next\_match}, the perl
pattern must be \emph{global} {\it i.e.}, it must have the modifier `g' at
the end ({\it e.g.}, {\tt m/a.b*/g}). Otherwise, {\tt next\_match} simply
fails and only one (first) match will be returned.  

\section{Bulk Matching}
XSB-perl interface also supports {\em bulk\/} matching through the
predicate {\tt bulk\_match/3}. Here, you get all
the substrings that match the patters at once, in a list, and then you can
process the matches as you wish. However, this does not give you full
access to submatches. More precisely, if you use parenthesized expressions,
then you get the list of non-null values in the variables \$1, \$2, etc.
If you do not use parenthesized regular expressions, you get the result of
the full match. For instance,
%%
\begin{verbatim}
  :- bulk_match('First 11.22 ; next: 3.4', 'm/(\d+)\.?(\d*)/g', Lst).
  Lst=[11,22,3,4]
  yes
  :- bulk_match('First 11.22 ; next: 3.4', 'm/\d+\.?\d*/g', Lst).
  Lst=[11.22,3.4]
  yes
\end{verbatim}
%%
{\tt bulk\_match/3} never fails. If there is no match, it returns an empty
list. 

Please note that you must specify `g' at the end of the pattern in order to
get something useful. This ia Perl thing! If you do not, instead of
returning a list of matches, Perl will think that you just want to test if
there is a match or not, and it will return \verb|[1]| or \verb|[]|,
depending on the outcome.

\section{String Substitution}
The last feature of the XSB-Perl interface is string substitution based on
pattern matching. This is achieved through the predicate
{\tt string\_substitute/3}: 

\begin{verbatim}
  :- perl_substitute(+String, +PerlSubstitutionExpr, -ResultString).
\end{verbatim}

We assume you are familiar with the syntax of Perl substitution
expressions. Here we just give an example of what kind of things are
possible:

\begin{verbatim}
  :- perl_substitute('this is fun', 's/(this) (is)(.*)/\2 \1\3?/', Str).
  Str=is this fun?
\end{verbatim}

\section{Unloading Perl}
Playing with Perl is nice, but this also means that both XSB and the Perl
interpreter  are loaded in the main memory. If you do not need Perl
for some time and memory is at premium, you can unload the Perl
interpreter:
%%
\begin{quote}
   {\tt :- unload\_perl. }
\end{quote}
%%
This predicate always succeeds. If you need Perl matching features later,
you can always come back to it: it is loaded automatically each time you
use a pattern matching or a string substitution predicate.

%%% Local Variables: 
%%% mode: latex
%%% TeX-master: "manual2"
%%% End: 
