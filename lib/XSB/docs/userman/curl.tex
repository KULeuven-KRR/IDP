
\chapter{{\tt curl}: The XSB Internet Access Package}

\begin{center}
  {\Large {\bf By Aneesh Ali}}
\end{center}



\section{Introduction}

The {\tt curl} package is an interface to the {\tt libcurl}  library,
which provides access to most of the standard Web protocols.
The supported protocols include
FTP, FTPS, HTTP, HTTPS, SCP, SFTP, TFTP, TELNET, DICT, LDAP, LDAPS, FILE, IMAP, SMTP, POP3 and RTSP. Libcurl supports SSL certificates, HTTP POST, HTTP PUT, FTP uploading, HTTP form based upload, proxies, cookies, user+password authentication (Basic, Digest, NTLM, Negotiate, Kerberos4), file transfer resume, http proxy tunneling etc. 


The {\tt curl} package accepts input in the form of URLs and Prolog
atoms. To load the {\tt curl} package, the user should type
%%
\begin{verbatim}
 ?- [curl].  
\end{verbatim}
%%
The {\tt curl} package is integrated with file I/O of XSB in a transparent
fashion and for many purposes Web pages can be treated just as yet another
kind of a file. We first explain how Web pages can be accessed
using the standard file I/O feature and then describe other predicates,
which provide a lower-level interface.

\section{Integration with File I/O}

The {\tt curl} package is integrated with XSB File I/O so that a web page
can be opened as any other file.Once a Web page is opened, it can be read
or written just like the a normal file.


\subsection{Opening a Web Document}

Web documents are opened by the usual predicates {\bf see/1}, {\bf open/3},
{\bf open/4}.

\begin{description}

  \index{\texttt{see/1}}
\item[see({\it url}({\it +Url}))]\mbox{}
  \index{\texttt{see/1}}
\item[see({\it url}({\it +Url,Options}))]\mbox{}
  \index{\texttt{open/3}}
\item[open({\it url}({\it +Url}), {\it +Mode}, {\it -Stream})]\mbox{}
  \index{\texttt{open/4}}
\item[open({\it url}({\it +Url}), {\it +Mode}, {\it -Stream}, {\it +Options})]\mbox{}
  \\

{\it Url} is an atom that specifies a URL.
\emph{Stream} is the file stream of the open file.
{\it Mode} can be
\paragraph{read} to create an input stream or
{\bf write},  to create an output stream.
For reading, the contents of the Web page are cached in a temporary file.
For writing, a temporary empty file is created. This file is posted to the
corresponding URL at closing.

The {\it Options} parameter is a list that controls loading. Members of that list can be of the following form:

  \begin{description}
  \item[{\bf redirect}{\bf (}{\it Bool}{\bf )}]\mbox{}
    \\
    Specifies the redirection option. The supported values are true and
    false. If true, any number of redirects is allowed. If false,
    redirections are ignored.
    The default is true.

  \item[{\bf secure}{\bf (}{\it CrtName}{\bf
    )}]\mbox{}
    \\
    Specifies the secure connections (https) option. \emph{CrtName} is the name of the file holding one or more certificates to verify the peer with. 

  \item[{\bf auth}{\bf (}{\it UserName, \it Password}{\bf )}]\mbox{}\\Sets the username and password basic authentication.

  \item[{\bf timeout}{\bf (}{\it Seconds}{\bf )}]\mbox{}\\Sets the maximum time in seconds that is allowed for the transfer operation.

  \item[{\bf user\_agent}{\bf (}{\it Agent}{\bf )}]\mbox{}\\Sets the User-Agent: header in the http request sent to the remote server.

  \end{description}


\end{description}

\subsection{Closing a Web Document}

Web documents opened by the predicates {\bf see/1}, {\bf open/3}, and {\bf
  open/4} above must be closed by the predicates {\bf close/2} or {\bf
  close/3}. The data written to the stream is first posted to the URL.
If that succeeds, the stream is closed. ???? And if it does not suceed????

\begin{description}

  \index{\texttt{close/3}}
\item[close({\it +Stream, +Source})]\mbox{}
  \index{\texttt{close/4}}
\item[close({\it +Stream, +Source, +Options})]\mbox{}
  \\
{\it Source} can be of the form {\tt url({\it {url}})}. Stream is a file stream. {\it Options} is a list of options supported normally for close.

\end{description}


\section{Low Level Predicates}

This section describes additional predicates provided by the {\tt
  curl} packages, which extend the functionality provided by the file I/O
integration.

\subsection{Loading web documents}

Web documents are loaded by the predicate {\bf load\_page/5}, which has
many options. The parameters of this predicate are described below.


\begin{description}
  \index{\texttt{load\_page/5}}
\item[load\_page({\it +Source, +Options, -Properties, -Content, -Warn})]\mbox{}
  \\
  {\it Source} can be of the form {\tt url({\it {url}})} or an atom
  \emph{url} (check!!!). 
  The document is returned in {\it Content}.
  {\it Warn} is bound to a (possibly empty) list of warnings generated during the process.

  {\it Properties} is bound to a list of properties of the document. They
  include {\it Directory name}, {\it File name}, {\it File suffix}, {\it
    Page size}, and {\it Page time}.
  The {\tt load\_page/5}  predicate caches a copy of the Web page that it
  fetched from the Web in a local file, which is specified by the above
  properties \emph{Directory name}, \emph{File name}, and \emph{File
    suffix}. The remaining two parameters indicate the size and the last
  modification time of the fetched Web page.
  The directory and the file name 
  The \emph{Options} parameter is the same as in the URL opening
  predicates. 

\end{description}

\subsection{Retrieve the properties of a web document}

The properties of a web document are loaded by the predicates {\bf
  url\_properties/3} and {\bf url\_properties/2}. 

\begin{description}
  \index{\texttt{url\_properties/3}}
\item[url\_properties({\it +Url, +Options, -Properties})]\mbox{}
  \\
  The {\it Options} and {\it Properties} are same as in {\bf load\_page/5}.
  \index{\texttt{url\_properties/2}}
\item[url\_properties({\it +Url, -Properties})]\mbox{}
  \\
  What are the default options???

\end{description}

\subsection{Encode Url}

Sometimes it is necessary to convert a URL string into something that can
be used, for example, as a file name. This is done by the following
predicate.

\begin{description}
  \index{\texttt{encode\_url/2}}
\item[encode\_url({\it +Source, -Result})]\mbox{}
  \\
{\it Source} has the form {\it url(url)} or an atom \emph{url}, where
\emph{url} is an atom.   (check!!!)
{\it Properties} is bound to a list of properties of the URL:
the encoded {\it Directory Name}, the encoded {\it File Name}, and the {\it
  Extension} of the URL.

\end{description}

\subsection{Obtaining the Redirection URL}

If the originally specified URL was redirected, the URL of the page that
was actually fetched by {\tt load\_page/5} can be found with the help of
the following predicate: 

\begin{description}
  \index{\texttt{get\_redir\_url/2}}
\item[get\_redir\_url({\it +Source, -UrlNew})]\mbox{}
  \\
  {\it Source} can be of the form {\tt url({\it {url}})}, {\tt file({\it {filename}})} or a string.

\end{description}

\section{Installation and configuration}

The {\tt curl} package of XSB requires that the {\tt libcurl} package is
installed.  For Windows, the {\tt libcurl} library files are included with
the installation. For Linux and Mac, the {\tt libcurl} and {\tt
  libcurl-dev} packages need to be installed using the distribution's
package manager. In some Linux distributions, {\tt libcurl-dev} might be
called {\tt libcurl-gnutls-dev} or {\tt libcurl-openssl-dev}.  In addition,
the release number might be attached. For instance, {\tt libcurl4} and {\tt
  libcurl4-openssl-dev}.

If a particular Linux distribution does not include the above packages and
for other Unix variants, the {\tt libcurl} package must be downloaded and
built manually. See
%% 
\begin{quote}
  \url{http://curl.haxx.se/download.html} 
\end{quote}
%% 
To configure {\tt curl} on Linux, Mac, or on some other Unix variant,
switch to the {\tt XSB/build} directory and type

%%
\begin{verbatim}
    cd XSB/packages/curl
    ./configure
    ./makexsb
\end{verbatim}
%%

%%% Local Variables: 
%%% mode: latex
%%% TeX-master: "manual2"
%%% End: 
