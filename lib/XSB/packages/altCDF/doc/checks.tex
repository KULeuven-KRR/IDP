\section{Semantic Checking} \label{sec:consist}

Enforcing the semantic axioms of Chapters~\ref{chap:type0} and
~\ref{chap:type1} is critical to developing ontologies.  In the first
place, enforcing a correct semantics is, in the long run, critical for
user confidence; in the second place, enforcing axioms at a system
level eases the coding of higher layers of CDF functionality and of
applications.  At the same time, semantic checking can slow down
performance if done redundantly or unnecessarily.

\index{contexts}
CDF provides several predicates for semantic checking, as well as
different methods to adjust the amount of checking done during
execution of system or user code.  Semantic checking in system code
can be adjusted by the predicicates~\pred{addCheckToContext/2}
and~\pred{removeCheckToContext/2}, which determine which checks are to
be performed at various {\em contexts} during the course of managing
an ontology.  The checking predicates themselves also can be called
directly, and users can even create their own contexts.  To explain
the consistency checking system, we first discuss the type of checking
predicates that are provided, then discuss system contexts, and
finally how a user can create his or her own contexts for application
code.

\subsection{Classes of Semantic Checking Predicates} \label{sec:checkpreds}

The semantic checking predicates may be categorized in different ways.
An obvious distinction is between Type-1 consistency checking and the
simpler semantic checks that make use of Type-0 axioms or of {\sc inh}
proofs.  The main Type-1 consistency checking predicate
is~\pred{checkIdConsistency/1} discussed in \refsec{sec:type1query}.
We discuss here the simpler semantic checking for (portions of) Type-0
Axioms.  Detailed information about these predicates are discussed in
\refsec{sec:checkingapi}.

Since well-sorting is a necessary condition for the consistency of
Type-0 (and Type-1) instances, certain of the checking predicatess
concern themselves with ensuring that given facts are well-sorted.
These include the predicates {\tt cdf\_check\_ground/1}, and {\tt
cdf\_check\_sorts/1}.  The first checks that a term is ground (modulo
the exception for identifiers whose component tag is
\component{cdfpt}); the second checks that an extensional fact is
well-sorted.

Other predicates perform checks to ensure that a CDF instance does not
have ``redundant'' extensional facts.  As notation, let us write an
extensional fact as $F\_ext\theta$ where $F\_ext$ is the predicate
skeleton (i.e. {\tt hasAttr\_ext(A,B,C)}, {\tt allAttr(A,B,C)} etc.)
and $\theta$ is the binding to the variables of the skeleton.  One
predicate that checks for redundancy is {\tt
cdf\_check\_implication($F\_ext\theta$)} which takes an extensional
fact as input and checks whether $F\theta$ holds in the current CDF
instance.  For example if the extensional fact were {\tt
hasAttr\_ext(A,B,C)$\theta$} the predicate would check whether {\tt
hasAttr(A,B,C)$\theta$} held.  A weaker predicate is
\pred{cdf\_check\_identity/1} which checks for $F\_ext\theta$
whether {\tt imnmed\_$F\theta$} holds in the current instance.  In
other words, if the extensional fact were {\tt
hasAttr\_ext(A,B,C)$\theta$} the predicate would check whether {\tt
immed\_hasAttr(A,B,C)$\theta$} held.  Working at a somewhat broader
level, the predicate {\tt check\_redundancies(Component)} performs the
same check as {\tt check\_identity/1} (in a somewhat optimized manner)
for each extensional fact in {\tt Component}, removing any redundant
facts (see Section~\ref{sec:components} for assignment of facts to
components).

If a check takes as input an extensional fact it is called a {\em
fact-level check}; if it takes as input a component tag it is called a
{\em component-level check}.  The type of input affects the contexts
where a check can be used, as we now describe.

%---------------------------------------------------------
\subsection{System Contexts}
\index{contexts}

The available system contexts for \version{} of CDF are listed.  

\begin{itemize}
\item Contexts where fact-level checks can be applied and are of type
{\tt factLevel}.
\begin{itemize}
\item \context{query}.  This context occurs when a query is made using the
Type-0 API.

\item \context{newExtTermSingle}.  This context occurs when an attempt is
made to add a new extensional fact to the CDF store by
\pred{newExtTerm/1}, as occurs for instance when facts are added using
the CDF editor.  

\item \context{newExtTermBatch} This context occurs when an attempt is
made to add a given extensional term to the CDF store during component
load by \pred{load\_component/3}, or file load by
\pred{load\_extensional\_facts/1}.

\item \context{retractallExtTermSingle}.  This context occurs when an attempt
is made to retract an extensional fact from the CDF store by
\pred{retractExtTerm/1}, as occurs for instance when facts are removed using
the CDF editor.

%\item \context{newIntRuleBatch} This context occurs when an attempt is
%made to add a given intensional rule to the CDF store during component
%load the predicate \pred{load\_intensional\_rules/1}.

%\item \context{newIntRuleSingle}.  This context occurs when an attempt is
%made to add a new intensional rule to the CDF store by
%\pred{newIntRule/1}.

%\item \context{retractIntRuleSingle}.  This context occurs when an attempt
%is made to retract an intensional rule from the CDF store by
%\pred{retractIntRule/1}, as occurs for instance when facts are removed
%using the CDF editor.
\end{itemize}

\item Contexts where component-level checks can be applied, and are of
type {\tt componentLevel}.
\begin{itemize}
\item \context{componentUpdate} This context occurs when the
predicate \pred{update\_all\_components/3} is called to save or move a
component.

\item \context{loadComponent} This context occurs  when the predicate
\pred{load\_component/3} is called to load a new component.
\end{itemize}
\end{itemize}

The checks are associated with a given context by default in
\file{cdf\_config.P} or in the user's \file{xsbrc.P} file.  However at
any point in a computation, a user may associate a given check to a
context or dissasociate the check from the context using
\pred{addCheckToContext/2} and~\pred{removeCheckFromContext/2}.

In addition, certain of the semantics checking predicates may be
useful outside of contexts, in order to check that a CDF fact is
properly ground or well-sorted.  To support this, there is a ``dummy''
context, {\tt usercall} that is displayed in warnings and errors in
these cases.

%---------------------------------------------------

\subsection{Adding User Contexts}

At an operational level, a context is executed whenever the predicate
{\tt apply\_checks(+Context,Argument)} is called, where {\tt
Argument} is either an extensional fact or a component tag, depending
on the type of the context.  This predicate determines which checks
are associated with {\tt Context} and ensures that all of the checks
are called.  Thus, the first step in adding a context to user code is
to determine what point (or points) checks should be made, and figure
out a name for the context.

In order to ensure that the checks are performed properly, the context
must be made known to CDF.  This is done by the predicate {\tt
addUserContext(+Context,+Type)} which informs CDF that a context of a
given type is added.  The context may be removed by {\tt
removeUserContext(+Context)}.

{\sc TLS add something about persistance for CDF flags}.

%---------------------------------------------------

\subsection{Semantic Checking Predicates} \label{sec:checkingapi}

\begin{description}
\ourpredmoditem{cdf\_check\_ground/1}{cdf\_checks}
{\tt cdf\_check\_ground(+Term)} requires that {\tt Term} be in the
domain \domain{isType0Term/1}, and checks that each argument {\tt A}
of {\tt Term} is a CDF identifier and that each identifier occurring
in {\tt A} is either ground or has the component tag
\component{cdfpt}.  If not, an instantiation error is thrown.

{\bf Exceptions} 

{\tt Domain Exception:} {\tt Term} is not in the domain {\tt
isType0Term/1}.

{\tt Instantiation Exception:} {\tt Term} has a non-ground identifier,
not in \component{cdfpt}.


% TLS: how about domain error?
\ourpredmodrptitem{cdf\_check\_sorts/2}{cdf\_checks}

\ourpredmoditem{cdf\_check\_sorts/1}{cdf\_checks}
{\tt cdf\_check\_sorts(+Context,+Term)} requires that {\tt Term} be in
the domain \domain{isType0Term/1}.  The predicate checks that each
argument {\tt A} of {\tt Term} is a CDF identifier that obeys the
sorting constraints of the instance axiom for {\tt Term}, as well as
the Downward Closure Axiom for Product Identifiers
(Axiom~\ref{ax:downcl}).  If {\tt Term} is not in the domain {\tt
isType0Term/1} or if it is not well-sorted, a warning is written out
to the XSB messages stream indicating that the predicate failed during
{\tt Context} and the predicate fails.  {\tt cdf\_check\_sorts(+Term)}
behaves similarly, with the context set to {\tt usercall}.

\ourpredmoditem{cdf\_check\_implication/1}{cdf\_checks}
{\tt cdf\_check\_implication(+Term)} requires that {\tt Term} be in
the domain \domain{isType0Term/1}.  The predicate silently fails if the
corresponding Type-0 predicate is derivable.  For instance, if {\tt
Term} were equal to {\tt hasAttr\_ext(C1,R1,C2)}, then the predicate
fails if {\tt hasAttr(C1,R1,C2)} holds in the current CDF instance.

\ourpredmoditem{cdf\_check\_identity/1}{cdf\_checks}
{\tt cdf\_check\_identity(+Term)} silently fails if {\tt ExtTerm} has
been asserted, or if the arguments of {\tt ExtTerm} can be derived by
a corresponding intensional rule; otherwise it succeeds.  For
instance, if {\tt ExtTerm} were equal to {\tt hasAttr\_ext(C1,R1,C2)},
then the predicate fails if {\tt ExtTerm} were already in the CDF
instance or if {\tt hasAttr\_int(C1,R1,C2)} were derivable.  As it
does not make use of inheritance rules, this predicate can much faster
than {\tt cdf\_check\_implication/1}

\ourpredmoditem{cdf\_check\_redundancies/3}{cdf\_checks}
If the mode in {\tt
cdf\_check\_redundancies(+Context,+Component,+Mode)} is {\tt retract},
this predicate backtracks through each extensional fact $F\_ext\theta$
(see Section~\ref{sec:checkpreds}) that are associated with {\tt
Component} and removes $F\_ext\theta$ if $F\theta$ is {\sc
inh}-implied by other extensional facts or intensional rules.
Otherwise, if {\tt Mode} is set to {\tt warn}, rather than removing
redundant facts, a warning is issued for each fact found to be
redundant.

\ourpredmoditem{checkComponentConsistency/1}{cdf\_checks}
{\tt checkComponentConsistency(+Component)} ensures that each class or
object identifier associated with {\tt Component} has a consistent
definition using the CDF theorem prover.

\ourpredmoditem{addCheckToContext/2}{cdf\_checks}
{\tt addCheckToContext(+PredIndicator,+Context)} ensures that the
check {\tt PredIndicator} (i.e. a term of the form {\tt F/N}) will be
performed when executing any context that unifies with {\tt Context}.
If {\tt PredIndicator} has already been added to {\tt Context}, {\tt
addCheckToContext/2} succeeds. 

{\bf Exceptions}:
    \begin{description}
    \item[{\tt instantiation\_error}]
    	{\tt PredIndiecator} or {\tt Context} is not instantiated at
	    time of call. 
    \item[{\tt misc\_error}]
    	{\tt Predindicator} is not a predicate indicator.
    \item[{\tt domain\_error}]
    	File {\tt Context} is not currently a context.
    \item[{\tt misc\_error}]
    	{\tt Context} and {\tt Predindicator} use different context types.
    \end{description}

\ourpredmoditem{removeCheckFromContext/2}{cdf\_checks}
{\tt removeCheckToContext(?PredIndicator,?Context)} ensures that any
checks that unify with {\tt PredIndicator} (i.e. a term of the form
{\tt F/N}) will {\em not} be no longer performed when executing {\tt
Context}.  If {\tt PredIndicator} is not associated with {\tt
Context}, {\tt removeCheckFromContext/2} succeeds.

\ourpredmoditem{addUserContext/2}{cdf\_checks}
{\tt addUserContext(+Context,+Type)} adds a context named {\tt
Context} of type {\tt Type}.  This action must be performed before any
checks can be added to {\tt Context}.  If {\tt Context} has already
been added with the same type the predicate succeeds, otherwise it
throws a permission error.

{\bf Exceptions}:
    \begin{description}
    \item[{\tt instantiation\_error}]
    	{\tt Context} or {\tt Type} is not instantiated at the time of call. 
    \item[{\tt domain\_error}]
    	File {\tt Type} is not in domain {\tt isContextType}
    \item[{\tt misc\_error}]
    	{\tt Context} is a system context.
    \item[{\tt misc\_error}]
    	{\tt Context} has already been added, with a type different
	    than {\tt Type}.
    \end{description}

\ourpredmoditem{removeUserContext/1}{cdf\_checks}
{\tt removeUserContext(?Context)} removes any contexts unifying with
{\tt Context} from CDF.  If {\tt Context} is not currently a context,
the predicate succeeds.

\ourpredmoditem{apply\_checks/2}{cdf\_checks}
{\tt apply\_checks(+Context,+Argument)} applys any checks that have
currently been added for {\tt Context} to {\tt Argument}.  If {\tt
Argument} is not of the right type, handling this error is left to the
checks for {\tt Context}, which may fail, emit warnings, throw errors,
etc.


\ourpreddomitem{isContextType/1}{cdf\_checks}
{\tt isContextType(?Type)} defines the context types for \version{} of
CDF.

\ourpredmoditem{currentContext/3}{cdf\_checks}
{\tt currentContext(?Context,?Mode,?Type)} defines the contexts for
the current CDF state, their ``mode'' (i.e. system or user) and their
type.
%
\end{description}
