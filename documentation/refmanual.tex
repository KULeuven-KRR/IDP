\documentclass{book}

\usepackage{xspace}
\usepackage{url}

\newcommand{\idp}{{\sc idp}\xspace}
%% Source code fragments
\usepackage[usenames,dvipsnames]{color}
\usepackage{listings}
\lstdefinelanguage{idp}
{
	morekeywords=[1]{vocabulary,theory,structure,procedure},
	morekeywords=[2]{type,isa},
	morekeywords=[3]{int,float,char,string},
	morecomment=[s]{/*}{*/},	
	morecomment=[l]{//}
}
\lstset{
	language=idp,
	tabsize=3,
	frame=none,
	basicstyle=\ttfamily,
	commentstyle=\color{Gray},
	keywordstyle=[1]\color{BrickRed}\bfseries,
	keywordstyle=[2]\color{OliveGreen}\bfseries,
	keywordstyle=[3]\color{Magenta}\bfseries,
}

\title{The \idp system reference manual}
\author{Johan Wittocx}
\begin{document}
\maketitle

%%%%%%%%%%%%%%%%%%%%%%%%%%%%%%%%%%%%%%%%%%%%%%%%%%%%%%%%%
\chapter{Comments}
%%%%%%%%%%%%%%%%%%%%%%%%%%%%%%%%%%%%%%%%%%%%%%%%%%%%%%%%%



%%%%%%%%%%%%%%%%%%%%%%%%%%%%%%%%%%%%%%%%%%%%%%%%%%%%%%%%%
\chapter{Vocabularies}
%%%%%%%%%%%%%%%%%%%%%%%%%%%%%%%%%%%%%%%%%%%%%%%%%%%%%%%%%

A vocabulary with name {\tt MyVoc} is declared by
\begin{lstlisting}
	vocabulary MyVoc {
		// contents of the vocabulary
	}
\end{lstlisting}



%%%%%%%%%%%%%%%%%%%%%%%%%%%%%%%%%%%%%%%%%%%%%%%%%%%%%%%%%
\chapter{Theories}
%%%%%%%%%%%%%%%%%%%%%%%%%%%%%%%%%%%%%%%%%%%%%%%%%%%%%%%%%

%%%%%%%%%%%%%%%%%%%%%%%%%%%%%%%%%%%%%%%%%%%%%%%%%%%%%%%%%
\chapter{Structures}
%%%%%%%%%%%%%%%%%%%%%%%%%%%%%%%%%%%%%%%%%%%%%%%%%%%%%%%%%

%%%%%%%%%%%%%%%%%%%%%%%%%%%%%%%%%%%%%%%%%%%%%%%%%%%%%%%%%
\chapter{Options}
%%%%%%%%%%%%%%%%%%%%%%%%%%%%%%%%%%%%%%%%%%%%%%%%%%%%%%%%%

%%%%%%%%%%%%%%%%%%%%%%%%%%%%%%%%%%%%%%%%%%%%%%%%%%%%%%%%%
\chapter{Procedures}
%%%%%%%%%%%%%%%%%%%%%%%%%%%%%%%%%%%%%%%%%%%%%%%%%%%%%%%%%

%%%%%%%%%%%%%%%%%%%%%%%%%%%%%%%%%%%%%%%%%%%%%%%%%%%%%%%%%
\section{Declaring a procedure}
%%%%%%%%%%%%%%%%%%%%%%%%%%%%%%%%%%%%%%%%%%%%%%%%%%%%%%%%%

A procedure with name {\tt MyProc} and arguments {\tt A1}, \ldots, {\tt An} is declared by 
\begin{lstlisting}
	procedure MyProc(A1,...,An) {
		// contents of the procedure
	}
\end{lstlisting}
Inside a procedure, any chunk of Lua code can be written. For Lua's reference manual, see \url{http://www.lua.org/manual/5.1/}. In the following, we assume that the reader is familiar with the basic concepts of Lua.

%%%%%%%%%%%%%%%%%%%%%%%%%%%%%%%%%%%%%%%%%%%%%%%%%%%%%%%%%
\section{\idp types}
%%%%%%%%%%%%%%%%%%%%%%%%%%%%%%%%%%%%%%%%%%%%%%%%%%%%%%%%%

Besides the standard types of variables available in Lua, the following extra types are available in \idp procedures.
\begin{description}
	\item[sort] A set of sorts with the same name. Can be used as a single sort if the set is a singleton.
	\item[predicate\_symbol] A set of predicates with the same name, but possibly with different arities. Can be used as a single predicate if the set is a singleton. If {\tt P} is a predicate\_symbol and {\tt n} an integer, then {\tt P/n} returns a predicate\_symbol containing all predicates in {\tt P} with arity {\tt n}. If {\tt s1}, \ldots, {\tt sn} are sorts, then {\tt P[s1,...,sn]} returns a predicate\_symbol containing all predicates $Q/n$ in {\tt P}, such that the $i$'th sort of $Q$ belongs to the set {\tt si}, for $1 \leq i \leq n$.
	\item[function\_symbol] A set of first-order functions with the same name, but possibly with different arities. Can be used as a single first-order function if the set is a singleton. If {\tt F} is a function\_symbol and {\tt n} an integer, then {\tt F/n:1} returns a function\_symbol containing all function in {\tt F} with arity {\tt n}. If {\tt s1}, \ldots, {\tt sn}, {\tt t} are sorts, then {\tt F[s1,...,sn:t]} returns a function\_symbol containing all functions $G/n$ in {\tt F}, such that the $i$'th sort of $F$ belongs to the set {\tt si}, for $1 \leq i \leq n$, and the output sort of $G$ belongs to {\tt t}.
	\item[symbol] A set of symbols of a vocabulary with the same name. Can be used as if it were a sort, predicate\_symbol, or function\_symbol.
	\item[vocabulary] A vocabulary. If {\tt V} is a vocabulary and {\tt s} a string, {\tt V[s]} returns the symbols in {\tt V} with name {\tt s}. 
	\item[compound] A domainelement of the form $F(d_1,\ldots,d_n)$, where $F$ is a first-order function and $d_1$, \ldots, $d_n$ are domain elements.
	\item[tuple] A tuple of domain elements. {\tt T[n]} returns the {\tt n}'th element in tuple {\tt T}.
	\item[predicate\_table] A table of tuples of domain elements.
	\item[predicate\_interpretation] An interpretation for a predicate. If {\tt T} is a predicate\_interpreation, then {\tt T.ct}, {\tt T.pt}, {\tt T.cf}, {\tt T.pf} return a predicate\_table containing, respectively, the certainly true, possibly true, certainly false, and possibly false tuples in {\tt T}. % HIER ONTBREEKT __newindex
	\item[function\_interpretation] An interpretation for a function. {\tt F.graph} returns the predicate\_interpretation of the graph associated to the function\_interpreation {\tt F}. % HIER ONTBREEKT __newindex en __call
	\item[structure] A first-order structure. To obtain the interpretation of a sort, singleton predicate\_symbol, or singleton function\_symbol {\tt symb} in structure {\tt S}, write {\tt S[symb]}.
	\item[theory] A logic theory.
	\item[options] A set of options.
	\item[namespace] A namespace.
	\item[overloaded] An overloaded object.
\end{description}

\end{document}
